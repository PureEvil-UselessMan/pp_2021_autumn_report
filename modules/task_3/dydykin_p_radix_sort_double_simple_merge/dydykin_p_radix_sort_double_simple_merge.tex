\documentclass[12pt]{report}
\usepackage{lingmacros}
\usepackage{tree-dvips}
\usepackage[utf8]{luainputenc}
\usepackage[english, russian]{babel}
\usepackage[T2A]{fontenc}
\usepackage{geometry}
\usepackage{listings}
\usepackage[pdftex]{hyperref}
\usepackage[14pt]{extsizes}
\usepackage{color}
\usepackage{enumitem}
\usepackage{multirow}
\usepackage{graphicx}
\usepackage{indentfirst}
\usepackage{amsmath}
\usepackage{tikz}
\usetikzlibrary{shapes.geometric}
\usetikzlibrary{arrows.meta,arrows}

\geometry{a4paper, top=2cm, bottom=3cm, left=2.5cm, right=1.5cm}

\setlength{\parskip}{0.5cm}
\setlist{nolistsep, itemsep=0.3cm, parsep=0pt}
\lstset{language=C++,
        basicstyle=\footnotesize,
        keywordstyle=\color{blue}\ttfamily,
        stringstyle=\color{red}\ttfamily,
        commentstyle=\color{green}\ttfamily,
        morecomment=[l][\color{magenta}]{\#},
        tabsize=4,
        breaklines=true,
        breakatwhitespace=true,
        title=\lstname}

\begin{document}
\begin{titlepage}
\begin{center}
    Министерство образования и науки Российской Федерации \\
    Федеральное государственное автономное образовательное учреждение высшего образования
\end{center}
\begin{center}
    \textbf{Национальный исследовательский Нижегородский государственный университет им. Н.И. Лобачевского} \\
\end{center}
\begin{center}
    \textbf{Институт информационных технологий, математики и механики}\\
\end{center}

\vspace{4em}

\begin{center}
    \textbf{\Large Отчет по лабораторной работе №3} \\
\end{center}

\begin{center}
    \textbf{Тема:} \\
    \textbf{\Large «Поразрядная сортировка для вещественных чисел (тип double) с простым слиянием»}
\end{center}

\vspace{4em}

\begin{flushright}
\begin{minipage}{0.55\textwidth}
\begin{flushleft}

\textbf{Выполнил:} \\
студент группы 381908-1 \\
Дыдыкин П.С. \\

\textbf{Проверил:} \\
доцент кафедры МОСТ, \\
кандидат технических наук \\
Сысоев А.В. \\
\end{flushleft}
\end{minipage}
\end{flushright}


\vspace{\fill}

\begin{center}
Нижний Новгород \\
2021
\end{center}

\end{titlepage}

\setcounter{page}{2}

\tableofcontents

\newpage

\section*{Введение}
\addcontentsline{toc}{section}{Введение}
\parСортировка или упорядочивание данных имеет очевидное практическое применение во многих областях - решение систем линейных уравнений, упорядочивание графов, базы данных. В современном мире число таких задач и их важность неуклонно растут. Решение таких задач на одном компьютере становится невозможным, поэтому встает потребность в распараллеливании программы, повышении эффективности параллельного алгоритма относительно последовательного. В данной лабораторной работе этот вопрос решается для алгоритма поразрядной сортировки вещественных чисел с простым слиянием, о чём речь пойдет далее.  

\newpage


\section*{Постановка задачи}
\addcontentsline{toc}{section}{Постановка задачи}
\parВ рамках данной лабораторной работы была поставлена задача сортировки вещественных чисел при помощи алгоритма поразрядной сортировки с применением простого слияния. Безусловно, для решения это задачи необходимо сперва изучить данный алгоритм, затем написать программный код, реализующий последовательную и параллельную обработку массива данных. Для проверки результатов и их корректности необходимо провести сравнение результирующих данных, а также провести сравнение эффективности путем замеров времени выполнения последовательного и параллельного алгоритма. Для выполнения поставленной задачи нам необходимо использовать программное обеспечение на основе технологии MPI и платформу тестирования программ Google Test.

\newpage


\section*{Описание алгоритма}
\addcontentsline{toc}{section}{Описание алгоритма}
\parВ нашей задаче используются вещественные числа, которые имеют в своем составе положительные (все, что слева от разделительной точки) и отрицательные (все, что справа от разделительной точки) разряды. Именно на использовании этих характеристик сортируемых элементов строится наш алгоритм, который состоит из нескольких этапов. Во-первых, необходимо определить максимальное количество положительных и отрицательных разрядов числа, по которым будем проводить сортировку. Затем мы должны разбить исходный массив на десять подмассивов, каждый из которых соотвествует одной из 10 арабских цифр (0 - 9). Далее проводим сортировку полученного массива путем выполнения поразрядной сортировки для каждого из разрядов, начиная с наименьшего и заканчивая наибольшим.

\newpage


\section*{Описание схемы распараллеливания}
\addcontentsline{toc}{section}{Описание схемы распараллеливания}
\parВ данной задаче мы используем следующую схему распараллеливания - каждый из процессов получается свою часть массива исходных данных и сортирует ее используя последовательный алгоритм сортировки, затем производится слияние всех частей массива в единый результирующий массив.
\parНа этапе подготовки массив исходных данных равномерно делится между процессами, участвующими в работе программы, при наличии остатка - он делится равномерно между теми процессами, ранг которых меньше самого значения остатка.
\parНа следующем этапе каждый процесс выполняет поразрядную сортировку полученного им подмассива. Таким образом получается, что каждый процесс обладает своим отсортированным массивом, теперь стоит задача слить все массивы в один. Для этого применяется простое слияние. Нулевой процесс выполняет слияние результирующего массива и массива, полученного от какого-либо процесса, результат слияния записывается также в результирующий массив. Предварительно в результирующем массиве уже лежит результат сортировки полученной части массива нулевым процессом. Само слияние производится путем последовательного сравнения элементов двух массивов и добавления на каждой из итераций наименьшего элемента в результирующий массив.


\parВ результате такого слияния на нулевом процессе получаем результирующий отсортированный массив.

\newpage


\section*{Описание программной реализации}
\addcontentsline{toc}{section}{Описание программной реализации}
\parПрограммная реализация алгоритма поразрядной сортировки представляет собой набор следующих функций:
\par 1. Функция генерации массива случайных данных (тип double)
\begin{lstlisting}
std::vector<double> Get_Random_Vector(int size);
\end{lstlisting}
\parОписание: реализует создание случайного массива данных в виде неотсортированного вектора вещественных чисел.

\par 2. Функция простого слияния
\begin{lstlisting}
 std::vector<double> Merge(const std::vector<double>& vec_l, const std::vector<double>& vec_r);
\end{lstlisting}
\parОписание: реализует простое слияние двух упорядоченных частей массива вещественных чисел.

\par 3. Функция поиска положительных разрядов числа
\begin{lstlisting}
int LeftOfThePoint(double number);
\end{lstlisting}
\parОписание: находит количество положительных разрядов вещественного числа.

\par 4. Функция поиска отрицательных разрядов числа
\begin{lstlisting}
int RightOfThePoint(double number);
\end{lstlisting}
\parОписание: находит количество отрицательных разрядов вещественного числа.

\par 5. Функция получения цифры из числа
\begin{lstlisting}
int GetDigit(double number, int radix);
\end{lstlisting}
\parОписание: находит и выдает цифру из заданного вещественного числа по заданному номеру разряда.

\par 6. Функция поразрядной сортировки
\begin{lstlisting}
std::vector<double> RadixSort(const std::vector<double>& vect, int rad);
\end{lstlisting}
\parОписание: реализует сортировку массива вещественных чисел по заданному номеру разряда.

\par 7. Функция последовательной поразрядной сортировки
\begin{lstlisting}
std::vector<double> Not_Parallel_Radix_Sort(const std::vector<double>& vect);
\end{lstlisting}
\parОписание: реализует последовательную поразрядную сортировку массива вещественных чисел.

\par 8. Функция параллельной поразрядной сортировки с простым слиянием
\begin{lstlisting}
std::vector<double> Parallel_Radix_Sort(const std::vector<double>& vec);
\end{lstlisting}
\parОписание: реализует параллельный алгоритм поразрядной сортировки для массива вещественных чисел с использованием простого слияния.

\newpage

\section*{Подтверждение корректности}
\addcontentsline{toc}{section}{Подтверждение корректности}
Для подтверждения корректности работы данной программы я разработал тесты с помощью фрэймфорка Google Test. Сначала мы проверяем отдельно работу последовательного и параллельного алгоримтов, сравнивая полученные значения с предварительно отсортированным массивом. Затем проводим сравнение полученных результатов в результате работы последовательного и параллельного алгоритма на одном и том же массиве вещественных чисел. Кроме того, производятся замеры скорости выполнения данных алгоримтов для оценки их эффективности.

\par Успешное прохождение всех тестов подтверждает корректность работы программы.
\newpage


\section*{Результаты экспериментов}
\addcontentsline{toc}{section}{Результаты экспериментов}
\parРезультаты эксперимента наиболее точно могут показать эффективность параллельного алгоритма поразрядной сортировки вещественных чисел перед последовательным. Исходные данные представлены в виде предварительно сгенерированного массива случаных вещественных числе из $10^{5}$ элементов, результаты эксперимента представлены в таблице 1:

\begin{table}[!h]
    \centering
    \begin{tabular}{ | l | l | l | l |}
    \hline
    \scriptsize{Кол-во процессов} & \scriptsize{Последовательный алгоритм (сек.)} & \scriptsize{Параллельный алгоритм (сек.)} & \scriptsize{Ускорение}  \\ \hline
    2   &   0.183629    &   0.103698    &   1.77081 \\ \hline
    3   &   0.181075    &   0.066269    &   2.73242  \\ \hline
    4   &   0.183130    &   0.052845    &   3.46542  \\ \hline
    6   &   0.201251    &   0.049539    &   4.06248  \\ \hline
    11   &   0.206116    &   0.032481    &   6.34574  \\ \hline
    \end{tabular}
    \caption{Результаты экспериментов}
    \end{table}

\parПо данным экспериментов можно отметить, что использование параллельной поразрядной
сортировки для вещественных (типа double) чисел простым слиянием более эффективно, чем использование последовательного алгоритма поразрядной сортировки.

\newpage


\section*{Заключение}
\addcontentsline{toc}{section}{Заключение}
\parВ данной лабораторной работе я познакомился с поразрядной сортировкой для вещественных чисел, а также рассмотрел не только последовательный алгоритм, но и параллельный. Были написаны и проведены тесты, наглядно показывающие эффективность параллельного алгоритма перед последовательным, а также подтверждающие корректность выполнения программы.
\parТаким образом, параллельный алгоритм лучше подойдет для решения сложных вычислительных задач с большим объемом данных, что доказывает его актуальность.
\newpage


\begin{thebibliography}{1}
\addcontentsline{toc}{section}{Список литературы}
\bibitem{Sidnev}Сиднев А.А., Сысоев А.В., Мееров И.Б. Образовательный комплекс
«Параллельные численные методы». Нижний Новгород, 2010.
\bibitem{Yakobovski}Якобовский М.В. Параллельные алгоритмы сортировки больших объемов данных, 2004, 2008.
\end{thebibliography}
\newpage

\section*{Приложение}
\addcontentsline{toc}{section}{Приложение}
В данном разделе приведен код, реализующий лабораторную работу №3 «Поразрядная сортировка для вещественных чисел (тип double) с простым слиянием».


\begin{lstlisting}
// radix_sort_double_simple_merge.h
// Copyright 2021 Dydykin Pavel
#ifndef MODULES_TASK_3_DYDYKIN_P_RADIX_SORT_DOUBLE_SIMPLE_MERGE_RADIX_SORT_DOUBLE_SIMPLE_MERGE_H_
#define MODULES_TASK_3_DYDYKIN_P_RADIX_SORT_DOUBLE_SIMPLE_MERGE_RADIX_SORT_DOUBLE_SIMPLE_MERGE_H_
#include <mpi.h>
#include <iostream>
#include <random>
#include <list>
#include <string>
#include <sstream>
#include <vector>

std::vector<double> Get_Random_Vector(int size);
int LeftOfThePoint(double num);
int RightOfThePoint(double num);
std::vector<double> Merge(const std::vector<double>& vec_left,
    const std::vector<double>& vec_right);
int GetDigit(double num, int radix);
std::vector<double> RadixSort(const std::vector<double>& vect, int rad);
std::vector<double> Not_Parallel_Radix_Sort(const std::vector<double>& vect);
std::vector<double> Parallel_Radix_Sort(const std::vector<double>& vec);

#endif  // MODULES_TASK_3_DYDYKIN_P_RADIX_SORT_DOUBLE_SIMPLE_MERGE_RADIX_SORT_DOUBLE_SIMPLE_MERGE_H_
\end{lstlisting}


\begin{lstlisting}
// radix_sort_double_simple_merge.cpp
// Copyright 2021 Dydykin Pavel
#include "radix_sort_double_simple_merge.h"


std::vector<double> Get_Random_Vector(int size) {
    std::mt19937 gen(time(0));
    std::uniform_real_distribution<> urd(0, 10000);
    std::vector<double> vector(size);
    for (int i = 0; i < static_cast<int>(vector.size()); i++) {
        vector[i] = urd(gen);
    }
    return vector;
}

std::vector<double> Merge(const std::vector<double>& vec_l,
    const std::vector<double>& vec_r) {
    std::vector<double> result((vec_l.size() + vec_r.size()));

    int vec_l_size = vec_l.size();
    int vec_r_size = vec_r.size();

    int i = 0, j = 0, k = 0;
    while (i < vec_l_size
        && j < vec_r_size) {
        if (vec_l[i] < vec_r[j]) {
            result[k] = vec_l[i];
            i++;
        } else {
            result[k] = vec_r[j];
            j++;
        }
        k++;
    }
    while (i < vec_l_size) {
        result[k] = vec_l[i];
        k++;
        i++;
    }
    while (j < vec_r_size) {
        result[k] = vec_r[j];
        k++;
        j++;
    }
    return result;
}


int LeftOfThePoint(double number) {
    int radix = 0;
    while (number > 1) {
        number = number / 10;
        radix++;
    }
    return radix;
}

int RightOfThePoint(double number) {
    std::ostringstream strs;
    strs << number;
    std::string str = strs.str();
    int size = str.size();
    if (str.find('.')) {
        int pos = str.find('.');
        int value = size - pos - 1;
        return value;
    } else {
        return 0;
    }
}

int GetDigit(double number, int radix) {
    if (radix > 0) {
        double mask = pow(10, radix);
        double tmp = number / mask;
        return static_cast<int>(tmp) % 10;
    }
    return  static_cast<int>(number * pow(10, -radix)) % 10;
}

std::vector<double> RadixSort(const std::vector<double>& vect, int rad) {
    std::vector<double> res;
    std::vector <std::vector<double>> radix(10);

    for (int i = 0; i < static_cast<int>(vect.size()); i++) {
        radix[GetDigit(vect[i], rad)].push_back(vect[i]);
    }
    for (int i = 0; i < static_cast<int>(radix.size()); ++i)
        for (int j = 0; j < static_cast<int>(radix[i].size()); ++j)
            res.push_back(radix[i][j]);
    return res;
}

std::vector<double> Not_Parallel_Radix_Sort(const std::vector<double>& vect) {
    int size = vect.size();
    int radixNegativeZero = 0;
    int maxRadixNegativeZero = RightOfThePoint(vect[0]);
    for (int i = 1; i < size; ++i) {
        radixNegativeZero = RightOfThePoint(vect[i]);
        if (radixNegativeZero > maxRadixNegativeZero) {
            maxRadixNegativeZero = radixNegativeZero;
        }
    }
    double max_elem = vect[0];
    for (int i = 1; i < size; i++) {
        if (vect[i] > max_elem) {
            max_elem = vect[i];
        }
    }
    int maxRadixPositiveZero = LeftOfThePoint(max_elem);
    std::vector<double> result(vect);
    for (int i = -maxRadixNegativeZero; i <= maxRadixPositiveZero; i++) {
        result = RadixSort(result, i);
    }
    return result;
}

std::vector<double> Parallel_Radix_Sort(const std::vector<double>& vec) {
    int ProcNum, ProcRank;
    MPI_Comm_size(MPI_COMM_WORLD, &ProcNum);
    MPI_Comm_rank(MPI_COMM_WORLD, &ProcRank);

    std::vector<double> result;

    int size = vec.size();
    std::vector<int> sendcounts(ProcNum);
    std::vector<int> displs(ProcNum);

    int count;

    if ((ProcRank < size % ProcNum))
        count = (size / ProcNum) + 1;
    else
        count = size / ProcNum;

    std::vector<double> recvbuf(count);

    if (recvbuf.size() == 0)
        recvbuf.resize(1);

    displs[0] = 0;

    for (int i = 0; i < ProcNum; i++) {
        if (i < (size % ProcNum))
            sendcounts[i] = (size / ProcNum) + 1;
        else
            sendcounts[i] = size / ProcNum;
        if (i > 0)
            displs[i] = (displs[i - 1] + sendcounts[i - 1]);
    }

    MPI_Scatterv(vec.data(), sendcounts.data(), displs.data(),
        MPI_DOUBLE, recvbuf.data(), count, MPI_DOUBLE, 0, MPI_COMM_WORLD);

    recvbuf = Not_Parallel_Radix_Sort(recvbuf);

    if (ProcRank != 0) {
        MPI_Send(recvbuf.data(), count, MPI_DOUBLE, 0, 0, MPI_COMM_WORLD);
    } else {
        result = recvbuf;
        MPI_Status status;
        for (int i = 1; i < ProcNum; i++) {
            recvbuf.resize(sendcounts[i]);
            MPI_Recv(recvbuf.data(), sendcounts[i], MPI_DOUBLE,
                i, MPI_ANY_TAG, MPI_COMM_WORLD, &status);
            result = Merge(result, recvbuf);
        }
    }
    return result;
}

\end{lstlisting}

\begin{lstlisting}
// main.cpp
// Copyright 2021 Dydykin Pavel
#include <gtest/gtest.h>
#include <algorithm>
#include "radix_sort_double_simple_merge.h"
#include <gtest-mpi-listener.hpp>

TEST(Radix_Sort, Test1_Not_Parallel_Radix_Sort_Double) {
    int ProcRank;
    MPI_Comm_rank(MPI_COMM_WORLD, &ProcRank);

    if (ProcRank == 0) {
        std::vector<double> v(50);
        v = Get_Random_Vector(50);

        std::vector<double> tmp = Not_Parallel_Radix_Sort(v);
        std::sort(v.begin(), v.end());
        ASSERT_EQ(tmp, v);
    }
    MPI_Barrier(MPI_COMM_WORLD);
}

TEST(Radix_Sort, Test2_Parallel_Radix_Sort_Double) {
    int ProcRank;
    MPI_Comm_rank(MPI_COMM_WORLD, &ProcRank);

    std::vector<double> v(50);
    std::vector<double> tmp(50);

    if (ProcRank == 0) {
        tmp = Get_Random_Vector(50);
        v = tmp;
        std::sort(tmp.begin(), tmp.end());
    }

    std::vector<double> sort = Parallel_Radix_Sort(v);

    if (ProcRank == 0) {
        ASSERT_EQ(sort, tmp);
    }
    MPI_Barrier(MPI_COMM_WORLD);
}

TEST(Radix_Sort, Test3_Not_Parallel_VS_Parallel_With_Small_Vector) {
    int ProcRank;
    MPI_Comm_rank(MPI_COMM_WORLD, &ProcRank);

    std::vector<double> v(50);
    std::vector<double> tmp(50);

    if (ProcRank == 0) {
        tmp = Get_Random_Vector(50);
        v = tmp;
        tmp = Not_Parallel_Radix_Sort(tmp);
    }

    std::vector<double> sort = Parallel_Radix_Sort(v);

    if (ProcRank == 0) {
        ASSERT_EQ(sort, tmp);
    }
    MPI_Barrier(MPI_COMM_WORLD);
}

TEST(Radix_Sort, Test4_Not_Parallel_VS_Parallel_With_Medium_Vector) {
    int ProcRank;
    MPI_Comm_rank(MPI_COMM_WORLD, &ProcRank);
    double t1, t2, t3, t4;

    std::vector<double> v(70);
    std::vector<double> not_p_vec, p_vec;

    if (ProcRank == 0) {
        v = Get_Random_Vector(70);
        t1 = MPI_Wtime();
        not_p_vec = Not_Parallel_Radix_Sort(v);
        t2 = MPI_Wtime();
    }

    if (ProcRank == 0) {
        t3 = MPI_Wtime();
    }

    p_vec = Parallel_Radix_Sort(v);

    if (ProcRank == 0) {
        t4 = MPI_Wtime();
    }

    if (ProcRank == 0) {
        ASSERT_EQ(p_vec, not_p_vec);
        std::cout << "NotParallel" << t2 - t1 << std::endl;
        std::cout << "Parallel" << t4 - t3 << std::endl;
        std::cout << "Effiency" << (t2 - t1) / (t4 - t3) << std::endl;
    }
    MPI_Barrier(MPI_COMM_WORLD);
}

TEST(Radix_Sort, Test5_Not_Parallel_VS_Parallel_With_Big_Vector_Effiency) {
    int ProcRank;
    MPI_Comm_rank(MPI_COMM_WORLD, &ProcRank);
    double t1, t2, t3, t4;

    std::vector<double> v(100);
    std::vector<double> not_p_vec, p_vec;

    if (ProcRank == 0) {
        v = Get_Random_Vector(100);
        t1 = MPI_Wtime();
        not_p_vec = Not_Parallel_Radix_Sort(v);
        t2 = MPI_Wtime();
    }

    if (ProcRank == 0) {
        t3 = MPI_Wtime();
    }

    p_vec = Parallel_Radix_Sort(v);

    if (ProcRank == 0) {
        t4 = MPI_Wtime();
    }

    if (ProcRank == 0) {
        ASSERT_EQ(p_vec, not_p_vec);
        std::cout << "NotParallel" << t2 - t1 << std::endl;
        std::cout << "Parallel" << t4 - t3 << std::endl;
        std::cout << "Effiency" << (t2 - t1) / (t4 - t3) << std::endl;
    }
    MPI_Barrier(MPI_COMM_WORLD);
}


int main(int argc, char** argv) {
    ::testing::InitGoogleTest(&argc, argv);
    MPI_Init(&argc, &argv);

    ::testing::AddGlobalTestEnvironment(new GTestMPIListener::MPIEnvironment);
    ::testing::TestEventListeners& listeners =
        ::testing::UnitTest::GetInstance()->listeners();

    listeners.Release(listeners.default_result_printer());
    listeners.Release(listeners.default_xml_generator());

    listeners.Append(new GTestMPIListener::MPIMinimalistPrinter);
    return RUN_ALL_TESTS();
}

\end{lstlisting}

\end{document}