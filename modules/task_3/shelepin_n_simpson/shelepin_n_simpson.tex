\documentclass{report}

\usepackage[warn]{mathtext}
\usepackage[T2A]{fontenc}
\usepackage[utf8]{luainputenc}
\usepackage[english, russian]{babel}
\usepackage[pdftex]{hyperref}
\usepackage[12pt]{extsizes}
\usepackage{listings}
\usepackage{color}
\usepackage{geometry}
\usepackage{enumitem}
\usepackage{multirow}
\usepackage{graphicx}
\usepackage{indentfirst}
\usepackage{amsmath}

\geometry{a4paper,top=2cm,bottom=2cm,left=2.5cm,right=1.5cm}
\setlength{\parskip}{0.5cm}
\setlist{nolistsep, itemsep=0.3cm,parsep=0pt}

\usepackage{listings}
\lstset{language=C++,
        basicstyle=\footnotesize,
		keywordstyle=\color{blue}\ttfamily,
		stringstyle=\color{red}\ttfamily,
		commentstyle=\color{green}\ttfamily,
		morecomment=[l][\color{red}]{\#}, 
		tabsize=4,
		breaklines=true,
  		breakatwhitespace=true,
  		title=\lstname,       
}

\makeatletter
\renewcommand\@biblabel[1]{#1.\hfil}
\makeatother

\begin{document}

\begin{titlepage}

\begin{center}
Министерство науки и высшего образования Российской Федерации
\end{center}

\begin{center}
Федеральное государственное автономное образовательное учреждение высшего образования \\
Национальный исследовательский Нижегородский государственный университет им. Н.И. Лобачевского
\end{center}

\begin{center}
Институт информационных технологий, математики и механики
\end{center}

\vspace{4em}

\begin{center}
\textbf{\LargeОтчет по лабораторной работе} \\
\end{center}
\begin{center}
\textbf{\Large«Вычисление многомерных интегралов с использованием многошаговой схемы 
(метод Симпсона)»} \\
\end{center}

\vspace{4em}

\newbox{\lbox}
\savebox{\lbox}{\hbox{text}}
\newlength{\maxl}
\setlength{\maxl}{\wd\lbox}
\hfill\parbox{7cm}{
\hspace*{5cm}\hspace*{-5cm}\textbf{Выполнил:} \\ студент группы 381906-2 \\ Шелепин Н.А.\\
\\
\hspace*{5cm}\hspace*{-5cm}\textbf{Проверил:}\\ доцент кафедры МОСТ, \\ кандидат технических наук \\ Сысоев А. В.\\
}
\vspace{\fill}

\begin{center} Нижний Новгород \\ 2021 \end{center}

\end{titlepage}

\setcounter{page}{2}

% Содержание
\tableofcontents
\newpage

% Введение
\section*{Введение}
\addcontentsline{toc}{section}{Введение}
\par Задачи численного интегрирования - распространённая задача, имеющая множество видов решения. Под численным интегрированием понимают набор численных методов для нахождения значения определённого интеграла. Основная идея большинства методов численного интегрирования состоит в замене подынтегральной функции на более простую, интеграл от которой легко вычисляется аналитически. Есть методы численного интегрирования, позволяющие получить значение определенного интеграла с требуемой степенью точности. Одним из таких методов является метод Симпсона (его еще называют методом парабол).
\par Формула Симпсона относится к приёмам численного интегрирования. Суть метода заключается в приближении подынтегральной функции на отрезке [a,b] интерполяционным многочленом второй степени, то есть приближение графика функции на отрезке параболой. 
\newpage

% Постановка задачи
\section*{Постановка задачи}
\addcontentsline{toc}{section}{Постановка задачи}
\par В данной лабораторной работе требуется реализовать последовательный и параллельные алгоритмы вычисления многомерных интегралов методом Симпсона.
\par Для оценки эффективности работы программы нужно произвести серию экспериментов, сравнивающих время выполнения последовательной и параллельной версии алгоритма, сравнить полученные результаты и сделать выводы.
\par Параллельный алгоритм должен быть реализован при помощи технологии MPI.
\newpage

% Описание алгоритма
\section*{Описание алгоритма}
\addcontentsline{toc}{section}{Описание алгоритма}
\par 
Формулой Симпсона называется интеграл от интерполяционного многочлена второй степени на отрезке [a,b]
$\displaystyle {\int_{a}^{b} f(x) \, dx }\approx {\int_{a}^{b} P_{2}(x) \, dx}=\frac  {b-a}{6}{\left(f(a)+4f\left({\frac  {a+b}{2}}\right)+f(b)\right)}$,
\par где ${\displaystyle f(a)}$, ${\displaystyle f((a+b)/2)}$ и ${\displaystyle f(b)}$ — значения функции в соответствующих точках (на концах отрезка и в его середине).
\par Для более точного вычисления интеграла, интервал [a,b] разбивают на N=2n элементарных отрезков одинаковой длины и применяют формулу Симпсона на составных отрезках. Каждый составной отрезок состоит из соседней пары элементарных отрезков. Значение исходного интеграла является суммой результатов интегрирования на составных отрезках:

${\displaystyle \int _{a}^{b}f(x)\,dx\approx {\frac {h}{3}}\left[f(x_{0})+2\sum _{j=1}^{n/2-1}f(x_{2j})+4\sum _{j=1}^{n/2}f(x_{2j-1})+f(x_{N})\right],}$
\par где ${\displaystyle h={\frac {b-a}{N}}}$ — величина шага, а ${\displaystyle x_{j}=a+jh}$ — чередующиеся границы и середины составных отрезков, на которых применяется формула Симпсона.
 \par Для вычисления многмерных интегралов воспользуемся многошаговой схемой.
\newpage

% Схема распараллеливания
\section*{Схема распараллеливания}
\addcontentsline{toc}{section}{Схема распараллеливания}
\par 
Основная идея реализации параллельного алгоритма состоит в параллельном вычислении локальных сумм по формуле Симпсона. Заданное количество интервалов распределяется между определенным количеством процессов. Каждый процесс
вычисляет значения подынтегральной функции для определенного числа отрезков, суммирует эти значения и отправляет получившиеся локальные суммы в процесс с рангом 0.
В 0-ом процессе все локальные суммы сначала складываются в глобальную сумму. После
результат будет умножаться на шаги разбиения отрезков интегрирования. В итоге получится приближенное значение искомого интеграла.
\newpage

% Описание программной реализации
\section*{Описание программной реализации}
\addcontentsline{toc}{section}{Описание программной реализации}
Программа состоит из заголовочного файла simpson.h и двух файлов исходного кода simpson.cpp и main.cpp.
\par В заголовочном файле находятся прототипы функций для последовательного и параллельного вычисления многомерных интегралов.
\par Функция для последовательного алгоритма:
\begin{lstlisting}
double getSequentialSimpson(
    const std::function<double(std::vector<double>)>& f,
    const std::vector<std::pair<double, double>>& limits,
    const std::vector<int>& n);
\end{lstlisting}
Входными параметрами функции является подынтегральная функция, массив пределов интегрирования, количество разбиений
\par Функция для параллельного
алгоритма:
\begin{lstlisting}
double getParallelSimpson(const std::function<double(std::vector<double>)>& f,
                          const std::vector<std::pair<double, double>>& limits,
                          const std::vector<int>& n);
\end{lstlisting}
Входные параметры данной функции совпадают с входными параметрами функции для последовательного алгоритма.
\par В файле исходного кода simpson.cpp содержится реализация функций, объявленных в заголовочном файле. В файле исходного кода main.cpp содержатся тесты для проверки корректности программы.
\newpage

% Подтверждение корректности
\section*{Подтверждение корректности}
\addcontentsline{toc}{section}{Подтверждение корректности}
Для подтверждения корректности работы данной программы с помощью фрэймфорка Google Test была разработана серия тестов. В каждом из тестов вычисляется значение контрольного  интеграла заданной размерности при помощи последовательного и параллельного алгоритмов, подсчитывается время работы обоих алгоритмов, находится ускорение делением времени работы последовательного алгоритма на время работы параллельного. Результаты вычисления интеграла последовательным и параллельным способом сравниваются между собой в пределах погрешности, после чего можно сделать выводы об эффективности программы.

\par Успешное прохождение разработанных мной тестов подтверждает корректность работы программы.
\newpage

% Результаты экспериментов
\section*{Результаты экспериментов}
\addcontentsline{toc}{section}{Результаты экспериментов}
Вычислительные эксперименты для оценки эффективности работы параллельного алгоритма проводились на ПК со следующими характеристиками:
\begin{itemize}
\item Процессор: AMD FX-8320E, 3.5 ГГц, количество ядер: 8;
\item Оперативная память: 16 ГБ DDR3, 1600  МГц;
\item Операционная система: Windows 10 Pro.
\end{itemize}

\par Результаты экспериментов представлены в Таблице 1.

\begin{table}[!h]
\caption{Результаты вычислительных экспериментов в тесте 4}
\centering
\begin{tabular}{| p{2cm} | p{3cm} | p{4cm} | p{2cm} |}
\hline
Количество процессов & Время работы последовательного алгоритма (в секундах) & Время работы параллельного алгоритма (в секундах) & Ускорение  \\[5pt]
\hline
1        & 1.02732        & 0.98748     & 1.04034       \\
2        & 1.16693        & 0.570968     & 2.04377       \\
3        & 1.00818        & 0.453759     & 2.22184       \\
4        & 0.987807        & 0.351503     & 2.81023       \\
6        & 1.11865        & 0.245628     & 4.55425       \\
8        & 0.979151        & 0.204774     & 4.78162	  \\
16        & 0.965454        & 0.243522     & 3.96454	  \\
\hline
\end{tabular}
\end{table}
По данным, полученным в результате экспериментов, можно сделать вывод о том,
что параллельный алгоритм работает быстрее, чем последовательный в 4.78 раз.

\newpage

% Заключение
\section*{Заключение}
\addcontentsline{toc}{section}{Заключение}
В ходе выполнения данной лабораторной работы были разработаны последовательный и параллельный алгоритмы вычисления многомерных интегралов методом Симпсона. После выполенения серии экспериментов можно сделать вывод о том, что параллельный алгоритм работает быстрее последовательного, что доказывает эффективность параллельного алгоритма по сравнению с последовательным.
\par Были разработаны и доведены до успешного выполнения тесты,  с использованием GoogleC++ Testing Framework, которые  подтвердиди корректность выполнения  программы.
\newpage

% Литература
\section*{Литература}
\addcontentsline{toc}{section}{Литература}
\begin{enumerate}
\item Гергель В. П. Теория и практика параллельных вычислений. – 2007.
\itemГергель В. П., Стронгин Р. Г. Основы параллельных вычислений для многопроцессорных вычислительных систем. – 2003.

\end{enumerate} 
\newpage

% Приложение
\section*{Приложение}
\addcontentsline{toc}{section}{Приложение}

simpson.cpp
\begin{lstlisting}
// Copyright 2021 Shelepin Nikita
#include <gtest/gtest.h>

#include <cmath>
#include <gtest-mpi-listener.hpp>

#include "./simpson.h"

TEST(SIMPSON_METHOD, TEST_FUNCTION_1) {
  int rank;
  MPI_Comm_rank(MPI_COMM_WORLD, &rank);

  if (rank == 0) {
    printf("  f(x,y) = x * x - 2 * y\n");
    fflush(stdout);
  }

  const std::function<double(std::vector<double>)> f =
      [](std::vector<double> vec) {
        double x = vec[0];
        double y = vec[1];
        return x * x - 2 * y;
      };

  std::vector<std::pair<double, double>> limits({{4, 10}, {1, 2}});
  std::vector<int> n({100, 100});

  double start = MPI_Wtime();
  double result = getParallelSimpson(f, limits, n);
  double end = MPI_Wtime();

  if (rank == 0) {
    double ptime = end - start;

    start = MPI_Wtime();
    double reference_result = getSequentialSimpson(f, limits, n);
    end = MPI_Wtime();
    double stime = end - start;

    std::cout << "  Sequential result: " << reference_result << std::endl;
    std::cout << "  Sequential time: " << stime << std::endl;
    std::cout << "  Parallel result: " << result << std::endl;
    std::cout << "  Parallel time: " << ptime << std::endl;
    std::cout << "  Speedup: " << stime / ptime << std::endl;

    double error = 0.0001;
    ASSERT_NEAR(result, reference_result, error);
  }
}

TEST(SIMPSON_METHOD, TEST_FUNCTION_2) {
  int rank;
  MPI_Comm_rank(MPI_COMM_WORLD, &rank);

  if (rank == 0) {
    printf("  f(x,y,z) = log10(2 * x * x) + sqrt(z) + 5 * y\n");
    fflush(stdout);
  }

  const std::function<double(std::vector<double>)> f =
      [](std::vector<double> vec) {
        double x = vec[0];
        double y = vec[1];
        double z = vec[2];
        return log10(2 * x * x) + sqrt(z) + 5 * y;
      };

  std::vector<std::pair<double, double>> limits({{4, 10}, {1, 2}, {2, 5}});
  std::vector<int> n({10, 10, 10});

  double result = getParallelSimpson(f, limits, n);

  if (rank == 0) {
    double reference_result = getSequentialSimpson(f, limits, n);
    double error = 0.0001;
    ASSERT_NEAR(result, reference_result, error);
  }
}

TEST(SIMPSON_METHOD, TEST_FUNCTION_3) {
  int rank;
  MPI_Comm_rank(MPI_COMM_WORLD, &rank);

  if (rank == 0) {
    printf("  f(x,y,z,t) = x + y + z + t\n");
    fflush(stdout);
  }

  const std::function<double(std::vector<double>)> f =
      [](std::vector<double> vec) {
        double x = vec[0];
        double y = vec[1];
        double z = vec[2];
        return x * y * z;
      };

  std::vector<std::pair<double, double>> limits({{4, 10}, {1, 2}, {4, 5}});
  std::vector<int> n({10, 5, 5});

  double result = getParallelSimpson(f, limits, n);

  if (rank == 0) {
    double reference_result = getSequentialSimpson(f, limits, n);
    double error = 0.0001;
    ASSERT_NEAR(result, reference_result, error);
  }
}

TEST(SIMPSON_METHOD, TEST_FUNCTION_4) {
  int rank;
  MPI_Comm_rank(MPI_COMM_WORLD, &rank);

  if (rank == 0) {
    printf("  f(x,y,z) = exp(x) - sqrt(10) * 5 * sin(y) + cos(-2 * z * z)\n");
    fflush(stdout);
  }

  const std::function<double(std::vector<double>)> f =
      [](std::vector<double> vec) {
        double x = vec[0];
        double y = vec[1];
        double z = vec[2];
        return exp(x) - sqrt(10) * 5 * sin(y) + cos(-2 * z * z);
      };

  std::vector<std::pair<double, double>> limits({{4, 10}, {1, 2}, {0, 5}});
  std::vector<int> n({10, 5, 5});

  double result = getParallelSimpson(f, limits, n);

  if (rank == 0) {
    double reference_result = getSequentialSimpson(f, limits, n);
    double error = 0.0001;
    ASSERT_NEAR(result, reference_result, error);
  }
}

TEST(SIMPSON_METHOD, TEST_FUNCTION_5) {
  int rank;
  MPI_Comm_rank(MPI_COMM_WORLD, &rank);

  if (rank == 0) {
    printf("  f(x,y,z,t) = cos(5 * x) + exp(y) + 2.9 * sin(z) - t * t\n");
    fflush(stdout);
  }

  const std::function<double(std::vector<double>)> f =
      [](std::vector<double> vec) {
        double x = vec[0];
        double y = vec[1];
        double z = vec[2];
        double t = vec[3];
        return cos(5 * x) + exp(y) + 2.9 * sin(z) - t * t;
      };

  std::vector<std::pair<double, double>> limits(
      {{4, 10}, {1, 2}, {1, 5}, {6, 10}});
  std::vector<int> n({5, 5, 3, 2});

  double start = MPI_Wtime();
  double result = getParallelSimpson(f, limits, n);
  double end = MPI_Wtime();

  if (rank == 0) {
    double ptime = end - start;

    start = MPI_Wtime();
    double reference_result = getSequentialSimpson(f, limits, n);
    end = MPI_Wtime();
    double stime = end - start;

    std::cout << "  Sequential result: " << reference_result << std::endl;
    std::cout << "  Sequential time: " << stime << std::endl;
    std::cout << "  Parallel result: " << result << std::endl;
    std::cout << "  Parallel time: " << ptime << std::endl;
    std::cout << "  Speedup: " << stime / ptime << std::endl;

    double error = 0.0001;
    ASSERT_NEAR(result, reference_result, error);
  }
}

int main(int argc, char** argv) {
  ::testing::InitGoogleTest(&argc, argv);
  MPI_Init(&argc, &argv);

  ::testing::AddGlobalTestEnvironment(new GTestMPIListener::MPIEnvironment);
  ::testing::TestEventListeners& listeners =
      ::testing::UnitTest::GetInstance()->listeners();

  listeners.Release(listeners.default_result_printer());
  listeners.Release(listeners.default_xml_generator());

  listeners.Append(new GTestMPIListener::MPIMinimalistPrinter);
  return RUN_ALL_TESTS();
}
\end{lstlisting}
main.cpp
\begin{lstlisting}
// Copyright 2021 Shelepin Nikita
#include <gtest/gtest.h>

#include <cmath>
#include <gtest-mpi-listener.hpp>

#include "./simpson.h"

TEST(SIMPSON_METHOD, TEST_FUNCTION_1) {
  int rank;
  MPI_Comm_rank(MPI_COMM_WORLD, &rank);

  if (rank == 0) {
    printf("  f(x,y) = x * x - 2 * y\n");
    fflush(stdout);
  }

  const std::function<double(std::vector<double>)> f =
      [](std::vector<double> vec) {
        double x = vec[0];
        double y = vec[1];
        return x * x - 2 * y;
      };

  std::vector<std::pair<double, double>> limits({{4, 10}, {1, 2}});
  std::vector<int> n({100, 100});

  double start = MPI_Wtime();
  double result = getParallelSimpson(f, limits, n);
  double end = MPI_Wtime();

  if (rank == 0) {
    double ptime = end - start;

    start = MPI_Wtime();
    double reference_result = getSequentialSimpson(f, limits, n);
    end = MPI_Wtime();
    double stime = end - start;

    std::cout << "  Sequential result: " << reference_result << std::endl;
    std::cout << "  Sequential time: " << stime << std::endl;
    std::cout << "  Parallel result: " << result << std::endl;
    std::cout << "  Parallel time: " << ptime << std::endl;
    std::cout << "  Speedup: " << stime / ptime << std::endl;

    double error = 0.0001;
    ASSERT_NEAR(result, reference_result, error);
  }
}

TEST(SIMPSON_METHOD, TEST_FUNCTION_2) {
  int rank;
  MPI_Comm_rank(MPI_COMM_WORLD, &rank);

  if (rank == 0) {
    printf("  f(x,y,z) = log10(2 * x * x) + sqrt(z) + 5 * y\n");
    fflush(stdout);
  }

  const std::function<double(std::vector<double>)> f =
      [](std::vector<double> vec) {
        double x = vec[0];
        double y = vec[1];
        double z = vec[2];
        return log10(2 * x * x) + sqrt(z) + 5 * y;
      };

  std::vector<std::pair<double, double>> limits({{4, 10}, {1, 2}, {2, 5}});
  std::vector<int> n({10, 10, 10});

  double result = getParallelSimpson(f, limits, n);

  if (rank == 0) {
    double reference_result = getSequentialSimpson(f, limits, n);
    double error = 0.0001;
    ASSERT_NEAR(result, reference_result, error);
  }
}

TEST(SIMPSON_METHOD, TEST_FUNCTION_3) {
  int rank;
  MPI_Comm_rank(MPI_COMM_WORLD, &rank);

  if (rank == 0) {
    printf("  f(x,y,z,t) = x + y + z + t\n");
    fflush(stdout);
  }

  const std::function<double(std::vector<double>)> f =
      [](std::vector<double> vec) {
        double x = vec[0];
        double y = vec[1];
        double z = vec[2];
        return x * y * z;
      };

  std::vector<std::pair<double, double>> limits({{4, 10}, {1, 2}, {4, 5}});
  std::vector<int> n({10, 5, 5});

  double result = getParallelSimpson(f, limits, n);

  if (rank == 0) {
    double reference_result = getSequentialSimpson(f, limits, n);
    double error = 0.0001;
    ASSERT_NEAR(result, reference_result, error);
  }
}

TEST(SIMPSON_METHOD, TEST_FUNCTION_4) {
  int rank;
  MPI_Comm_rank(MPI_COMM_WORLD, &rank);

  if (rank == 0) {
    printf("  f(x,y,z) = exp(x) - sqrt(10) * 5 * sin(y) + cos(-2 * z * z)\n");
    fflush(stdout);
  }

  const std::function<double(std::vector<double>)> f =
      [](std::vector<double> vec) {
        double x = vec[0];
        double y = vec[1];
        double z = vec[2];
        return exp(x) - sqrt(10) * 5 * sin(y) + cos(-2 * z * z);
      };

  std::vector<std::pair<double, double>> limits({{4, 10}, {1, 2}, {0, 5}});
  std::vector<int> n({10, 5, 5});

  double result = getParallelSimpson(f, limits, n);

  if (rank == 0) {
    double reference_result = getSequentialSimpson(f, limits, n);
    double error = 0.0001;
    ASSERT_NEAR(result, reference_result, error);
  }
}

TEST(SIMPSON_METHOD, TEST_FUNCTION_5) {
  int rank;
  MPI_Comm_rank(MPI_COMM_WORLD, &rank);

  if (rank == 0) {
    printf("  f(x,y,z,t) = cos(5 * x) + exp(y) + 2.9 * sin(z) - t * t\n");
    fflush(stdout);
  }

  const std::function<double(std::vector<double>)> f =
      [](std::vector<double> vec) {
        double x = vec[0];
        double y = vec[1];
        double z = vec[2];
        double t = vec[3];
        return cos(5 * x) + exp(y) + 2.9 * sin(z) - t * t;
      };

  std::vector<std::pair<double, double>> limits(
      {{4, 10}, {1, 2}, {1, 5}, {6, 10}});
  std::vector<int> n({5, 5, 3, 2});

  double start = MPI_Wtime();
  double result = getParallelSimpson(f, limits, n);
  double end = MPI_Wtime();

  if (rank == 0) {
    double ptime = end - start;

    start = MPI_Wtime();
    double reference_result = getSequentialSimpson(f, limits, n);
    end = MPI_Wtime();
    double stime = end - start;

    std::cout << "  Sequential result: " << reference_result << std::endl;
    std::cout << "  Sequential time: " << stime << std::endl;
    std::cout << "  Parallel result: " << result << std::endl;
    std::cout << "  Parallel time: " << ptime << std::endl;
    std::cout << "  Speedup: " << stime / ptime << std::endl;

    double error = 0.0001;
    ASSERT_NEAR(result, reference_result, error);
  }
}

int main(int argc, char** argv) {
  ::testing::InitGoogleTest(&argc, argv);
  MPI_Init(&argc, &argv);

  ::testing::AddGlobalTestEnvironment(new GTestMPIListener::MPIEnvironment);
  ::testing::TestEventListeners& listeners =
      ::testing::UnitTest::GetInstance()->listeners();

  listeners.Release(listeners.default_result_printer());
  listeners.Release(listeners.default_xml_generator());

  listeners.Append(new GTestMPIListener::MPIMinimalistPrinter);
  return RUN_ALL_TESTS();
}
\end{lstlisting}

\end{document}