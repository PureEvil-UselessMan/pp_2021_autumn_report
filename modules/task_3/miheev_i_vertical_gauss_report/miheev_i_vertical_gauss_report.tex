\documentclass{report}

\usepackage[T2A]{fontenc}
\usepackage[utf8]{luainputenc}
\usepackage[english, russian]{babel}
\usepackage[pdftex]{hyperref}
\usepackage[14pt]{extsizes}
\usepackage{listings}
\usepackage{color}
\usepackage{geometry}
\usepackage{enumitem}
\usepackage{multirow}
\usepackage{graphicx}
\usepackage{indentfirst}
\usepackage{amsmath}

\geometry{a4paper,top=2cm,bottom=3cm,left=2cm,right=1.5cm}
\setlength{\parskip}{0.5cm}
\setlist{nolistsep, itemsep=0.3cm,parsep=0pt}

\lstset{language=C++,
		basicstyle=\footnotesize,
		keywordstyle=\color{blue}\ttfamily,
		stringstyle=\color{red}\ttfamily,
		commentstyle=\color{green}\ttfamily,
		morecomment=[l][\color{magenta}]{\#}, 
		tabsize=4,
		breaklines=true,
  		breakatwhitespace=true,
  		title=\lstname,       
}

\makeatletter
\renewcommand\@biblabel[1]{#1.\hfil}
\makeatother

\begin{document}

\begin{titlepage}

\begin{center}
Министерство науки и высшего образования Российской Федерации
\end{center}

\begin{center}
Федеральное государственное автономное образовательное учреждение высшего образования \\
Национальный исследовательский Нижегородский государственный университет им. Н.И. Лобачевского
\end{center}

\begin{center}
Институт информационных технологий, математики и механики
\end{center}

\vspace{4em}

\begin{center}
\textbf{\Large Отчет по лабораторной работе} \\
\end{center}
\begin{center}
\textbf{\Large «Линейная фильтрация изображений (вертикальное разбиение). Ядро Гаусса 3x3»} \\
\end{center}

\vspace{4em}

\newbox{\lbox}
\savebox{\lbox}{\hbox{text}}
\newlength{\maxl}
\setlength{\maxl}{\wd\lbox}
\hfill\parbox{7cm}{
\hspace*{5cm}\hspace*{-5cm}\textbf{Выполнил:} \\ студент группы 381906-3 \\ Михеев И. А.\\
\\
\hspace*{5cm}\hspace*{-5cm}\textbf{Проверил:}\\ доцент кафедры МОСТ, \\ кандидат технических наук \\ Сысоев А. В.\\
}
\vspace{\fill}

\begin{center} Нижний Новгород \\ 2021 \end{center}

\end{titlepage}

\setcounter{page}{2}

% Содержание
\tableofcontents
\newpage

% Введение
\section*{Введение}
\addcontentsline{toc}{section}{Введение}
Размытие изображений часто применяется для выделения общих черт объектов при распознавании образов, создания различных постэффектов в компьютерных играх, сокрытия конфиденциальной информации, уменьшения шума и т.д. Существует несколько способов размытия изображений, самым распространенным из которых является размытие Гаусса. Данный способ основан на использовании функции Гаусса, которая совпадает с функцией плотности вероятности нормального распределения. Данная функция используется для построения ядра свертки, которое применяется к каждому пикселю изображения для его размытия. \par
В данной лабораторной работе будет рассмотрена линейная фильтрация изображения с применением ядра Гаусса размером 3х3. Особенностью данной реализации будет то, что фильтр будет применяться по столбцам изображения. В программе реализованы как последовательная, так и параллельная версии линейной фильтрации, что позволит сравнить эффективность методов.
\newpage

% Постановка задачи
\section*{Постановка задачи}
\addcontentsline{toc}{section}{Постановка задачи}
Для успешного выполнения лабораторной работы необходимо реализовать последовательную и параллельную реализации линейной фильтрации изображения с ядром Гаусса 3х3 с проходом по столбцам, проверить корректность работы алгоритмов, провести эксперименты для оценки эффективности параллельной реализации. По полученным результатам сделать выводы. \par 
Для реализации параллельной версии программы используется библиотека межпроцессорного взаимодействия MPI. Для проверки корректности работы алгоритмов требуется использовать Google C++ Testing Framework.
\newpage

% Описание алгоритма
\section*{Описание алгоритма}
\addcontentsline{toc}{section}{Описание алгоритма}
Для того, чтобы применить размытие Гаусса к изображению, необходимо: \par
\begin{itemize}
\item Рассчитать весовые коэффициенты ядра свёртки.
\item Применить матрицу  размером 3x3  свертки для пикселя изображения и области вокруг него, для этого значение каждого пикселя из этой области умножается на значение из соответствующей ячейки в ядре.
\item Полученный результат нормируется.
\item Новое значение пикселя численно равно линейной комбинации значений яркости пикселов в окне фильтрации с коэффициентами матрицы весов фильтра.
\item Повторить предыдущие действия для каждого пикселя изображения.
\end{itemize}
\par Возможны несколько вариантов свертки на границах изображения, но в данной лабораторной работе граничные пиксели в матрице свертки не учитывались.
\newpage

% Схема распараллеливания
\section*{Схема распараллеливания}
\addcontentsline{toc}{section}{Схема распараллеливания}
\par Для эффективного распараллеливания алгоритма линейной фильтрации необходимо передать каждому процесу равное количество столбцов изображения, которые он должен обработать, а также по одному столбцу справа и слева от них. Корневой процесс рассылает одинаковое количество столбцов всем процессам (рассылаемые области пересекаются, т.к. нужно отправлять еще и граничные с областью обработки элементы), при этом сам корневой процесс так же учавствует в обработке, дополнительно принимая остаток столбцов. Каждый процесс создает ядро Гаусса и с помощью него проводит свертку на своей части изображения. Корневой процесс собирает результаты фильтрации и создает из одно изображение. Полученное изображение является результатом работы программы.
\newpage

% Описание программной реализации
\section*{Описание программной реализации}
\addcontentsline{toc}{section}{Описание программной реализации}

Алгоритм расчета коэффициентов ядра посредством функции Гаусса:
\begin{lstlisting}
float* getKernel()
\end{lstlisting}
Функция возвращает матрицу свертки в виде массива типа float.

Алгоритм применяющий ядро свертки к пикселю с координатами x,y:
\begin{lstlisting}
float calcNewPixelColor(const int* image, int x, int y, int width, int height, const float* kernel)
\end{lstlisting}
Входные параметры:
\begin{itemize}
\item image - изображение для обработки в виде одномерного массива int
\item x - координата пикселя по оси абсцисс
\item y - координата пикселя по оси ординат
\item height - высота изображения
\item width - ширина изображения
\item kernel - коэффициенты матрицы свертки в виде массива
\end{itemize}
Возвращает новое значение пикселя.

Последовательная реализация размытия Гаусса:
\begin{lstlisting}
int* getSequentialGauss(const int* image, int width, int height)
\end{lstlisting}
Входные параметры:
\begin{itemize}
\item image - изображение для обработки в виде одномерного массива int
\item height - высота изображения
\item width - ширина изображения
\end{itemize}
Возвращает изображение после применения размытия Гаусса (последовательная реализация).

Параллельная реализация размытия Гаусса:
\begin{lstlisting}
int* getParallelGauss(const int* image, int width, int height)
\end{lstlisting}
Входные параметры:
\begin{itemize}
\item image - изображение для обработки в виде одномерного массива int
\item height - высота изображения
\item width - ширина изображения
\end{itemize}
Возвращает изображение после применения размытия Гаусса (параллельная реализация).
\newpage

% Подтверждение корректности
\section*{Подтверждение корректности}
\addcontentsline{toc}{section}{Подтверждение корректности}
Для подтверждения корректности работы программы представлен набор тестов, разработанных с использованием Google C++ Testing Framework.
\par Набор тестов представляет собой случайные изображения разных размеров с целью проверки корректности вычислений (сравнивается изображение, полученное с помощью последовательной фильтрации, с изображением, полученным с помощью параллельной фильтрации), а также эффективность алгоритмов по отношению друг к другу (сравнивается время выполнения последовательной и параллельной реализации).
\par Успешное прохождение всех тестов означает корректность работы программного комплекса, а разнице во времени выполнения с последовательной реализацией можно судить об эффективности параллельной реализации.
\newpage

% Результаты экспериментов
\section*{Результаты экспериментов}
\addcontentsline{toc}{section}{Результаты экспериментов}
Эксперименты для оценки эффективности параллельной реализации линейной фильтрации с использованием ядра Гаусса проводились на машине со следующей конфигурацией:

\begin{itemize}
\item Процессор: Intel(R) Core(TM) i7-6700 CPU @ 3.40GHz, количество ядер: 8;
\item Оперативная память: 8 ГБ;
\item ОС: Microsoft Windows 10;
\end{itemize}
https://ru.overleaf.com/project/61cd8c1f5996084e278b40d5
\par Для проведения экспериментов производилась фильтрация изображений размером 10000x10000. 
\par Результаты экспериментов представлены в таблице.

\begin{table}[!h]
\caption{Результаты вычислительных экспериментов}
\centering
\begin{tabular}{lllll}
Процессы & Параллельно & Последовательно & Ускорение  \\
1        & 4.09115        & 3.812       & 0.93176     \\
2        & 3.40964        & 3.85394     & 1.13030     \\  
3        & 3.39951        & 3.9608      & 1.16511     \\
4        & 2.92732        & 3.90841     & 1.33514     \\
5        & 2.75154        & 3.83014     & 1.3922       \\
6        & 2.42149        & 3.92431     & 1.62061     \\
7        & 2.25451        & 3.84617     & 1.70599     \\
8        & 1.94417        & 3.78874     & 1.9487
\end{tabular}
\end{table}

\par По замерам времени экспериментов, подтверждено, что параллельный случай работает быстрее, чем последовательный.
\newpage

% Заключение
\section*{Заключение}
\addcontentsline{toc}{section}{Заключение}
В результате лабораторной работы были реализованы последовательная и параллельная реализации линейной фильтрации изображения с использованием ядра Гаусса, были разработаны успешно выполнены тесты, созданные для данного программного проекта с использованием Google C++ Testing Framework и необходимые для подтверждения корректности работы программы. Параллельная реализация оказалась эффективней последовательной, что подтверждает успешное выполнение лабораторной работы.
\newpage

% Список литературы
\begin{thebibliography}{1}
\addcontentsline{toc}{section}{Список литературы}
\bibitem{parallel} Алгоритмы быстрого размытия изображений, основанные на фильтре Гаусса [Электронный ресурс] // URL: \url {https://www.cs.vsu.ru/ipmt-conf/conf/2021/works/} //
\bibitem{Algolist} Algolist: Алгоритмы, методы, исходники [Электронный ресурс] // URL: \url {https://en.wikipedia.org/wiki/Gaussian_filter} //
\end{thebibliography}
\newpage

% Приложение
\section*{Приложение}
\addcontentsline{toc}{section}{Приложение}

\begin{lstlisting}
// Copyright 2021 Miheev Ivan
#ifndef MODULES_TASK_3_MIHEEV_I_VERTICAL_GAUSS_VERTICAL_GAUSS_H_
#define MODULES_TASK_3_MIHEEV_I_VERTICAL_GAUSS_VERTICAL_GAUSS_H_
int* getRandomImage(int width, int height);
int* getSequentialGauss(const int* image, int width, int height);
int* getParallelGauss(const int* image, int width, int height);

#endif  // MODULES_TASK_3_MIHEEV_I_VERTICAL_GAUSS_VERTICAL_GAUSS_H_
\end{lstlisting}

\begin{lstlisting}
// Copyright 2021 Miheev Ivan

#include "../../../modules/task_3/miheev_i_vertical_gauss/vertical_gauss.h"

#include <mpi.h>

#include <ctime>
#include <iostream>
#include <random>

int* getRandomImage(int width, int height) {
  std::mt19937 gen(time(0));
  std::uniform_int_distribution<> uid(0, 255);
  int* img = new int[width * height];
  for (int i = 0; i < width * height; i++) {
    img[i] = uid(gen);
  }
  return img;
}

int clamp(int value, int max, int min) {
  if (value > max) return max;
  if (value < min) return min;
  return value;
}

float* getKernel() {
  int size = 3;
  int len = size*size;

  float* kernel = new float[len];
  float sigma = 3.f;
  float norm = 0;
  int signed_radius = 1;

  for (int x = -signed_radius; x <= signed_radius; x++)
    for (int y = -signed_radius; y <= signed_radius; y++) {
      std::size_t idx = (x + 1) * size + (y + 1);
      kernel[idx] = std::exp(-(x * x + y * y) / (sigma * sigma));
      norm += kernel[idx];
    }
  for (int i = 0; i < len; i++) {
    kernel[i] /= norm;
  }

  return kernel;
}

float calcNewPixelColor(const int* image, int x, int y, int width, int height,
                        const float* kernel) {
  float returnC = 0;

  for (int i = -1; i <= 1; i++) {
    for (int j = -1; j <= 1; j++) {
      size_t idx = (i + 1) * 3 + j + 1;
      int pixel = image[x + j + (y + i) * width];
      returnC += pixel * kernel[idx];
    }
  }
  return clamp(returnC, 255, 0);
}


int* getSequentialGauss(const int* image, int width, int height) {
  int* new_image = new int[(width - 2) * (height - 2)];

  const float* kernel = getKernel();

  for (int i = 1; i < height - 1; i++) {
    for (int j = 1; j < width - 1; j++) {
      new_image[j - 1 + (i - 1) * (width - 2)] =
          calcNewPixelColor(image, j, i, width, height, kernel);
    }
  }
  return new_image;
}

int* getParallelGauss(const int* image, int width, int height) {
  int ProcNum, ProcRank;
  MPI_Comm_size(MPI_COMM_WORLD, &ProcNum);
  MPI_Comm_rank(MPI_COMM_WORLD, &ProcRank);

  int delta = (width - 2) / ProcNum;
  int rem = (width - 2) % ProcNum;

  if (delta) {
    int *counts = new int[ProcNum], *displs = new int[ProcNum];
    for (int i = 0; i < ProcNum; i++) {
      if (i == 0) {
        counts[i] = delta + rem + 2;
        displs[i] = 0;
      } else {
        counts[i] = delta + 2;
        displs[i] = delta * i + rem;
      }
    }

    int* local_matrix;

    int temp = delta;
    if (ProcRank == 0) temp += rem;

    local_matrix = new int[(temp + 2) * height];

    for (int i = 0; i < height; i++) {
      MPI_Scatterv(image + i * width, counts, displs, MPI_INT,
                   local_matrix + i * (temp + 2), temp + 2, MPI_INT, 0,
                   MPI_COMM_WORLD);
    }

    local_matrix = getSequentialGauss(local_matrix, temp + 2, height);

    int* global_matrix = nullptr;

    for (int i = 0; i < ProcNum; i++) {
      counts[i] -= 2;
    }

    if (ProcRank == 0) global_matrix = new int[(width - 2) * (height - 2)];

    for (int i = 0; i < height - 2; i++) {
      MPI_Gatherv(local_matrix + i * temp, temp, MPI_INT,
                  global_matrix + i * (width - 2), counts, displs, MPI_INT, 0,
                  MPI_COMM_WORLD);
    }

    return global_matrix;
  } else {
    if (ProcRank == 0) {
      return getSequentialGauss(image, width, height);
    } else {
      return nullptr;
    }
  }
}
\end{lstlisting}

\begin{lstlisting}
// Copyright 2021 Miheev Ivan
#include <gtest/gtest.h>

#include <gtest-mpi-listener.hpp>
#include "./vertical_gauss.h"

TEST(Vertical_Gauss_MPI, Test_width_priority_2) {
  int rank;
  MPI_Comm_rank(MPI_COMM_WORLD, &rank);

  int* image = nullptr;
  const int width = 10000, height = 10000;

  if (rank == 0) {
    image = getRandomImage(width, height);
  }
  double start = MPI_Wtime();
  int* new_image_parallel = getParallelGauss(image, width, height);
  double end = MPI_Wtime();

  if (rank == 0) {
    double time1 = end - start;
    start = MPI_Wtime();
    int* reference_new_image = getSequentialGauss(image, width, height);
    end = MPI_Wtime();
    double time2 = end - start;
    std::cout << time1 << " " << time2 << " " << time2 / time1 << std::endl;
    for (int i = 0; i < (width - 2) * (height - 2); i += (width - 2)) {
      ASSERT_EQ(new_image_parallel[i], reference_new_image[i]);
    }
  }
}

// Main
int main(int argc, char** argv) {
  ::testing::InitGoogleTest(&argc, argv);
  MPI_Init(&argc, &argv);

  ::testing::AddGlobalTestEnvironment(new GTestMPIListener::MPIEnvironment);
  ::testing::TestEventListeners& listeners =
      ::testing::UnitTest::GetInstance()->listeners();

  listeners.Release(listeners.default_result_printer());
  listeners.Release(listeners.default_xml_generator());

  listeners.Append(new GTestMPIListener::MPIMinimalistPrinter);
  return RUN_ALL_TESTS();
}
\end{lstlisting}
\end{document}