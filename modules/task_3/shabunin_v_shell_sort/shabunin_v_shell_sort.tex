\documentclass{report}
\usepackage[T2A]{fontenc}
\usepackage{amsmath,amsthm,amssymb}
\usepackage{mathtext}
\usepackage{enumitem}
\usepackage{graphicx}
\usepackage[utf8]{luainputenc}
\usepackage[english,russian]{babel}
\usepackage{lmodern}
\usepackage{geometry}
\usepackage{listings}
\usepackage{alltt}
\usepackage{relsize}
\usepackage{color}
\usepackage{microtype}
\usepackage{indentfirst}
\microtypesetup{expansion=false}

\geometry{a4paper,top=2cm,bottom=2cm,left=2.5cm,right=1.5cm}
\setlength{\parskip}{0.5cm}
\setlist{nolistsep, itemsep=0.3cm,parsep=0pt}

\usepackage{listings}
\lstset{language=C++,
        basicstyle=\footnotesize,
		keywordstyle=\color{blue}\ttfamily,
		stringstyle=\color{red}\ttfamily,
		commentstyle=\color{green}\ttfamily,
		morecomment=[l][\color{red}]{\#}, 
		tabsize=4,
		breaklines=true,
  		breakatwhitespace=true,
  		title=\lstname,       
}

\makeatletter
\renewcommand\@biblabel[1]{#1.\hfil}
\makeatother

\begin{document}
	
	\begin{titlepage}
		
		\begin{center}
        Министерство науки и высшего образования Российской Федерации
        \end{center}

        \begin{center}
        Федеральное государственное автономное образовательное учреждение высшего образования \\
        Национальный исследовательский Нижегородский государственный университет им. Н.И. Лобачевского
        \end{center}

        \begin{center}
        Институт информационных технологий, математики и механики
        \end{center}
		
		\vspace{4em}
		
		\begin{center}
			\textbf{\LargeОтчет по лабораторной работе} \\
		\end{center}
		\begin{center}
			\textbf{\Large«Сортировка Шелла с простым слиянием»} \\
		\end{center}
		
		\vspace{4em}
		
		\newbox{\lbox}
		\savebox{\lbox}{\hbox{text}}
		\newlength{\maxl}
		\setlength{\maxl}{\wd\lbox}
		\hfill\parbox{5cm}{
			\hspace*{5cm}\hspace*{-5cm}\textbf{Выполнил:} \\ студент группы 381806-3 \\ Шабунин В. С.\\
			\\
			\hspace*{5cm}\hspace*{-5cm}\textbf{Проверил:}\\ доцент кафедры МОСТ, \\ кандидат технических наук \\ Сысоев А. В.\\}
		\vspace{\fill}
		
		\begin{center} Нижний Новгород \\ 2021 \end{center}
		
	\end{titlepage}
	
	\setcounter{page}{2}
	
	% Содержание
	\tableofcontents
	\newpage
	

% Введение
\section*{Введение}
\addcontentsline{toc}{section}{Введение}
\par Сортировка была названа в честь её изобретателя — Дональда Шелла, который опубликовал этот алгоритм в 1959 году. Сортировка Шелла — алгоритм сортировки, являющийся усовершенствованным вариантом сортировки вставками. Идея метода Шелла состоит в сравнении элементов, стоящих не только рядом, но и на определённом расстоянии друг от друга. Иными словами — это сортировка вставками с предварительными «грубыми» проходами.
\par Сортировки, имеющие сложность \(N^2\) хорошо справляются с небольшими наборами данных (и, к тому же, обычно не требуют использования дополнительной памяти), а сортировки со сложностью исполнения \(N(\log N)\) подойдут для больших объемов.

\par Сортировка Шелла - одна из самых простых в понимании и реализации среди быстрых
сортировок. Эта сортировка используется в ядре Linux, и в, как минимум, в одной библиотеке
С++ её код используется для стандартной функции qsort().

\par Обычно сортировка Шелла часто даже быстрее теоретически более быстрых по порядку методов и начинает чуть уступать им лишь, при обработке довольно больших массивов (порядка 10-в миллионов элементов). Этой сортировке совершенно не нужна дополнительная память и она стабильно ведёт себя для самых разных вариантов заполнения данных, выгодно отличаясь этим от быстрых сортировок.
\newpage

% Постановка задачи
\section*{Постановка задачи}
\addcontentsline{toc}{section}{Постановка задачи}
В данной лабораторной работе требуется
\begin{itemize}
\item реализовать последовательный и параллельный метод сортировки Шелла на С++ с использованием библиотеки MPI.
\item составить тесы и сравнить эффективность алгоритмов
\end{itemize}
\newpage
% Описание алгоритма
\section*{Описание алгоритма}
\addcontentsline{toc}{section}{Описание алгоритма}
\par При сортировке Шелла сначала сравниваются и сортируются между собой значения, стоящие один от другого на некотором расстоянии d. После этого процедура повторяется для некоторых меньших значений d, а завершается сортировка Шелла упорядочиванием элементов при d=1 (то есть обычной cортировкой вставками). Эффективность сортировки Шелла в определённых случаях обеспечивается тем, что элементы «быстрее» встают на свои места (в простых методах сортировки, например, пузырьковой, каждая перестановка двух элементов уменьшает количество инверсий в списке максимум на 1, а при сортировке Шелла это число может быть больше). 
\par В моей реализации 1 значение d равно половине длинны массива. Каждый следующий шаг уменьшает это значение на 1
\newpage
% Описание схемы распараллеливания
\section*{Описание схемы распараллеливания}
\addcontentsline{toc}{section}{Описание схемы распараллеливания}
На начальном этапе имеем сгенерированный некоторый массив, который хранится в 0 процессе.

Если общее количество процессов больше 1 и размер исходного массива больше 1, начинаем процесс параллельной сортировки Шелла, а именно:
\begin{enumerate} 
\item Выбираем начальный шаг (в каждом случае он кратен 2ум)
\item Заводим внешний цикл while, который будет работать до тех пор, пока шаг нашей сортировки не станет = 0.
\item Внутри цикла while заведём цикл do{}while для корректного разбиения проходов по процессам.
\item Теперь с текущим шагом проходим по массиву несколько раз и каждый из наборов отправляем в отдельный процесс
\item В каждом из процессов сортируем данные и отправляем их обратно в 0 процесс для сортировки слиянием.

\end{enumerate}
\newpage
% Описание программной реализации
\section*{Описание программной реализации}
\addcontentsline{toc}{section}{Описание программной реализации}
Программа состоит из заголовочного файла shell\_sort.h и двух файлов исходного кода shell\_sort.cpp и main.cpp.

\par Функция для генирации вектора:
\begin{lstlisting}
std::vector<int> getRandomVector(int sz)
\end{lstlisting}
 Генерирует вектор заданной длинны со случайными данными типа int
 
 \par Функция для сортировки слиянием:
\begin{lstlisting}
std::vector<int> merge(std::vector<int>* v1, std::vector<int>* v2)
\end{lstlisting}
 Принимает в себя отсортированные вектора(сортировкой Шелла) и производит сортировку слиянием

\par Функция для последовательного алгоритма:
\begin{lstlisting}
void getSequentialShellSort(std::vector<int>* vec)
\end{lstlisting}
 Принимает в себя вектор и производит сортировку

\par Функция для параллельного
алгоритма:
\begin{lstlisting}
void getParallelShellSort(std::vector<int>* vec)
\end{lstlisting}
Принимает в себя вектор и делит его.
\par В файле исходного кода shell\_sort.cpp содержится реализация функций, объявленных в заголовочном файле. В файле исходного кода main.cpp содержатся тесты для проверки корректности программы.
\newpage
% Подтверждение корректности
\section*{Подтверждение корректности}
\addcontentsline{toc}{section}{Подтверждение корректности}
Для подтверждения корректности работы были созданы 5 Google test
\begin{alltt}
TEST(SHELL_SORT, TEST_SIZE_128) {}
TEST(SHELL_SORT, TEST_SIZE_123) {}
TEST(SHELL_SORT, TEST_SIZE_1024) {}
TEST(SHELL_SORT, TEST_SIZE_1569) {}
TEST(SHELL_SORT, TEST_SIZE_2048) {}
\end{alltt}
В каждом тесте меняется длинна генерируемого вектора.
\newpage
% Результаты экспериментов
\section*{Результаты экспериментов}
\addcontentsline{toc}{section}{Результаты экспериментов}
Эксперименты проводились на ПК со следующей конфигурацией:
\begin{itemize}
\item Операционная система: Windows 10 Pro версия 20H2
\item Процессор: AMD Ryzen 9 5900X  3.70 GHz
\item ОЗУ 32 гб
\item Версия Visual Studio: 2022
\end{itemize}
\begin{table}[h]
  \caption{Количество процессов: 2}
  \centering
\begin{tabular}{| p{2cm} | p{3cm} | p{4cm} | p{2cm} |}
\hline
Количество элементов & Время работы последовательного алгоритма (в секундах) & Время работы параллельного алгоритма (в секундах) & Во сколько раз эффективнее  \\[5pt]
\hline
2048        & 0.00018        & 0.0002644     & 0.6807      \\
2345        & 0.0002356       & 0.0001067     & 2.208       \\
4096        & 0.0002187       & 0.0006766      & 3.093       \\
4567        & 0.0008421        & 0.0002784     & 3.021      \\
8192        & 0.0002606        & 0.0007016    & 3.714       \\
\hline
\end{tabular}
\end{table}

\begin{table}[h]
  \caption{Количество процессов: 4}
  \centering
\begin{tabular}{| p{2cm} | p{3cm} | p{4cm} | p{2cm} |}
\hline
Количество элементов & Время работы последовательного алгоритма (в секундах) & Время работы параллельного алгоритма (в секундах) & Во сколько раз эффективнее  \\[5pt]
\hline
2048        & 0.0001771      & 0.0004474     & 0.39      \\
2345      & 0.0002285       & 0.0000703     & 3.25       \\
4096        & 0.0007092        & 0.0001328      & 5.34       \\
4567        & 0.0009182        & 0.0001432     & 6.41      \\
8192        & 0.0026171        & 0.0003128     & 8.36       \\
\hline
\end{tabular}
\end{table}

\begin{table}[h]
  \caption{Количество процессов: 8}
  \centering
\begin{tabular}{| p{2cm} | p{3cm} | p{4cm} | p{2cm} |}
\hline
Количество элементов & Время работы последовательного алгоритма (в секундах) & Время работы параллельного алгоритма (в секундах) & Во сколько раз эффективнее  \\[5pt]
\hline
2048        & 0.0001771      & 0.0007569     & 0.23      \\
2345      & 0.000225       & 0.0000724     & 3.1       \\
4096        & 0.0007111        & 0.0000981      & 7.24       \\
4567        & 0.0008518        & 0.0000865     & 9.84      \\
8192        & 0.0001822        & 0.0001822     & 13.83       \\
\hline
\end{tabular}
\end{table}

\begin{table}[h]
  \caption{Количество процессов: 16}
  \centering
\begin{tabular}{| p{2cm} | p{3cm} | p{4cm} | p{2cm} |}
\hline
Количество элементов & Время работы последовательного алгоритма (в секундах) & Время работы параллельного алгоритма (в секундах) & Во сколько раз эффективнее  \\[5pt]
\hline
2048        & 0.0001848      & 0.0011575     & 0.15      \\
2345      & 0.000226       & 0.0000802     & 2.82       \\
4096        & 0.000101        & 0.000101      & 15.38       \\
4567        & 0.000966        & 0.000126     & 7.61      \\
8192        & 0.0002629        & 0.000204     & 12.84       \\
\hline
\end{tabular}
\end{table}

\begin{table}[h]
  \caption{Количество процессов: 24}
  \centering
\begin{tabular}{| p{2cm} | p{3cm} | p{4cm} | p{2cm} |}
\hline
Количество элементов & Время работы последовательного алгоритма (в секундах) & Время работы параллельного алгоритма (в секундах) & Во сколько раз эффективнее  \\[5pt]
\hline
2048        & 0.0001794      & 0.001584     & 0.11      \\
2345      & 0.0002333       & 0.0000747     & 3.12       \\
4096        & 0.00066        & 0.0000969      & 6.81       \\
4567        & 0.0008113        & 0.0001305     & 6.21      \\
8192        & 0.000208        & 0.0002083     & 14.18       \\
\hline
\end{tabular}
\end{table}




\newpage

% Выводы из результатов экспериментов
\section*{Выводы из результатов экспериментов}
\addcontentsline{toc}{section}{Выводы из результатов экспериментов}
По данным экспериментам можно понять что алгоритм становится более эффективным при большом колличестве элементов. Это связанно с его стабильностью при увеличении элементов.

\newpage
% Заключение
\section*{Заключение}
\addcontentsline{toc}{section}{Заключение}
Реализованные алгоритмы Шелла позволяют оценить разность в производительности последовательного и параллельного подхода, а также изучить конкретные примеры их реализаций.
\newpage

% Список литературы

\addcontentsline{toc}{section}{Список литературы}
\begin{enumerate}
\item https://habr.com/ru/post/467473/
\item https://ru.wikipedia.org/wiki/%D0%A1%D0%BE%D1%80%D1%82%D0%B8%D1%80%D0%BE%D0%B2%D0%BA%D0%B0_%D0%A8%D0%B5%D0%BB%D0%BB%D0%B0


\item С. В. Скворцов, Т. А. Пюрова. ПАРАЛЛЕЛЬНЫЕ АЛГОРИТМЫ СОРТИРОВКИ ДАННЫХ
И ИХ РЕАЛИЗАЦИЯ НА ПЛАТФОРМЕ CUDA.

\item MPI documentation, https://www.mpich.org/static/docs/v3.2/www3/
\end{enumerate}
\newpage
% Приложение
\section*{Приложение}
\addcontentsline{toc}{section}{Приложение}
 	\subsection*{shell\_sort.h}
 	\begin{lstlisting}
// Copyright 2021 Shabunin Vladislav
#ifndef MODULES_TASK_3_SHABUNIN_V_SHELL_SORT_SHELL_SORT_H_
#define MODULES_TASK_3_SHABUNIN_V_SHELL_SORT_SHELL_SORT_H_

#include <vector>

std::vector<int> getRandomVector(int sz);
void getSequentialShellSort(std::vector<int>* vec);
void getParallelShellSort(std::vector<int>* arr);

#endif  // MODULES_TASK_3_SHABUNIN_V_SHELL_SORT_SHELL_SORT_H_


\end{lstlisting}
shell\_sort.cpp
\begin{lstlisting}
// Copyright 2021 Shabunin Vladislav
#include "../../../modules/task_3/shabunin_v_shell_sort/shell_sort.h"

#include <mpi.h>

#include <algorithm>
#include <random>
#include <string>
#include <vector>

std::vector<int> getRandomVector(int sz) {
  std::random_device dev;
  std::mt19937 gen(dev());
  std::vector<int> vec(sz);
  for (int i = 0; i < sz; i++) {
    vec[i] = gen() % 100;
  }
  return vec;
}

std::vector<int> merge(std::vector<int>* v1, std::vector<int>* v2) {
  int i = 0, j = 0, k = 0;
  int n1 = (*v1).size(), n2 = (*v2).size();
  std::vector<int> result(n1 + n2);

  while (i < n1 && j < n2)
    if ((*v1)[i] < (*v2)[j]) {
      result[k] = (*v1)[i];
      i++;
      k++;
    } else {
      result[k] = (*v2)[j];
      j++;
      k++;
    }
  if (i == n1) {
    while (j < n2) {
      result[k] = (*v2)[j];
      j++;
      k++;
    }
  } else {
    while (i < n1) {
      result[k] = (*v1)[i];
      i++;
      k++;
    }
  }
  return result;
}


void getSequentialShellSort(std::vector<int>* vec) {
  int i, j, increment, temp, size = (*vec).size();
  increment = 3;
  while (increment > 0) {
    for (i = 0; i < size; i++) {
      j = i;
      temp = (*vec)[i];
      while ((j >= increment) && ((*vec)[j - increment] > temp)) {
        (*vec)[j] = (*vec)[j - increment];
        j = j - increment;
      }
      (*vec)[j] = temp;
    }
    if (increment / 2 != 0)
      increment = increment / 2;
    else if (increment == 1)
      increment = 0;
    else
      increment = 1;
  }
}

void getParallelShellSort(std::vector<int>* vec) {
  std::vector<int> chunk;
  std::vector<int> other;
  int m = (*vec).size(), n = (*vec).size();
  int rank, size;
  int delta;
  int i;
  int step;
  MPI_Status status;

  MPI_Comm_rank(MPI_COMM_WORLD, &rank);
  MPI_Comm_size(MPI_COMM_WORLD, &size);

  if (rank == 0) {
    int r;
    delta = n / size;
    r = n % size;

    if (r != 0) {
      (*vec).resize(n + size - r);
      for (i = n; i < n + size - r; i++) (*vec)[i] = 0;
      delta = delta + 1;
    }

    MPI_Bcast(&delta, 1, MPI_INT, 0, MPI_COMM_WORLD);
    chunk.resize(delta);
    MPI_Scatter((*vec).data(), delta, MPI_INT, chunk.data(), delta, MPI_INT, 0,
                MPI_COMM_WORLD);
    getSequentialShellSort(&chunk);
  } else {
    MPI_Bcast(&delta, 1, MPI_INT, 0, MPI_COMM_WORLD);
    chunk.resize(delta);
    MPI_Scatter((*vec).data(), delta, MPI_INT, chunk.data(), delta, MPI_INT, 0,
                MPI_COMM_WORLD);
    getSequentialShellSort(&chunk);
  }

  step = 1;

  while (step < size) {
    if (rank % (2 * step) == 0) {
      if (rank + step < size) {
        MPI_Recv(&m, 1, MPI_INT, rank + step, 0, MPI_COMM_WORLD, &status);
        other.resize(m);
        MPI_Recv(other.data(), m, MPI_INT, rank + step, 0, MPI_COMM_WORLD,
                 &status);
        chunk = merge(&chunk, &other);
        delta = delta + m;
        fflush(stdout);
      }
    } else {
      int near = rank - step;
      MPI_Send(&delta, 1, MPI_INT, near, 0, MPI_COMM_WORLD);
      MPI_Send(chunk.data(), delta, MPI_INT, near, 0, MPI_COMM_WORLD);
      break;
    }
    step = step * 2;
  }
  if (rank == 0) {
    (*vec) = chunk;
    if (n % size) {
      for (i = n; i < n + size - n % size; i++) (*vec).erase((*vec).begin());
    }
  }
}
\end{lstlisting}
main.cpp
\begin{lstlisting}
// Copyright 2021 Shabunin Vladislav
#include <gtest/gtest.h>

#include <vector>
#include <gtest-mpi-listener.hpp>

#include "./shell_sort.h"

TEST(SHELL_SORT, TEST_SIZE_128) {
  int rank, num;
  MPI_Comm_rank(MPI_COMM_WORLD, &rank);
  MPI_Comm_size(MPI_COMM_WORLD, &num);

  std::vector<int> vec, vec1, vec2;
  int vec_size = 128;
  double start, end;

  if (rank == 0) {
    vec = getRandomVector(vec_size);
    vec1 = vec;
    // for (int elem : vec) {
    //  std::cout << elem << " ";
    //}
    // std::cout << std::endl;
    start = MPI_Wtime();
  }
  getParallelShellSort(&vec1);
  if (rank == 0) {
    end = MPI_Wtime();
    double ptime = end - start;
    std::cout << "P time " << ptime << std::endl;
    vec2 = vec;
    start = MPI_Wtime();
    getSequentialShellSort(&vec2);
    end = MPI_Wtime();
    double stime = end - start;
    std::cout << "S time " << stime << std::endl;
    std::cout << "Speedup " << stime / ptime << std::endl;

    std::cout << vec1.size() << " " << vec2.size() << std::endl;
    for (int i = 0; i < vec_size; i++) {
      ASSERT_EQ(vec1[i], vec2[i]);
    }
  }
}

TEST(SHELL_SORT, TEST_SIZE_123) {
  int rank, num;
  MPI_Comm_rank(MPI_COMM_WORLD, &rank);
  MPI_Comm_size(MPI_COMM_WORLD, &num);

  std::vector<int> vec, vec1, vec2;
  int vec_size = 123;
  double start, end;

  if (rank == 0) {
    vec = getRandomVector(vec_size);
    vec1 = vec;
    // for (int elem : vec) {
    //  std::cout << elem << " ";
    //}
    // std::cout << std::endl;
    start = MPI_Wtime();
  }
  getParallelShellSort(&vec1);
  if (rank == 0) {
    end = MPI_Wtime();
    double ptime = end - start;
    std::cout << "P time " << ptime << std::endl;
    vec2 = vec;
    start = MPI_Wtime();
    getSequentialShellSort(&vec2);
    end = MPI_Wtime();
    double stime = end - start;
    std::cout << "S time " << stime << std::endl;
    std::cout << "Speedup " << stime / ptime << std::endl;

    std::cout << vec1.size() << " " << vec2.size() << std::endl;
    for (int i = 0; i < vec_size; i++) {
      ASSERT_EQ(vec1[i], vec2[i]);
    }
  }
}

TEST(SHELL_SORT, TEST_SIZE_1024) {
  int rank, num;
  MPI_Comm_rank(MPI_COMM_WORLD, &rank);
  MPI_Comm_size(MPI_COMM_WORLD, &num);

  std::vector<int> vec, vec1, vec2;
  int vec_size = 1024;
  double start, end;

  if (rank == 0) {
    vec = getRandomVector(vec_size);
    vec1 = vec;
    // for (int elem : vec) {
    //  std::cout << elem << " ";
    //}
    // std::cout << std::endl;
    start = MPI_Wtime();
  }
  getParallelShellSort(&vec1);
  if (rank == 0) {
    end = MPI_Wtime();
    double ptime = end - start;
    std::cout << "P time " << ptime << std::endl;
    vec2 = vec;
    start = MPI_Wtime();
    getSequentialShellSort(&vec2);
    end = MPI_Wtime();
    double stime = end - start;
    std::cout << "S time " << stime << std::endl;
    std::cout << "Speedup " << stime / ptime << std::endl;

    std::cout << vec1.size() << " " << vec2.size() << std::endl;
    for (int i = 0; i < vec_size; i++) {
      ASSERT_EQ(vec1[i], vec2[i]);
    }
  }
}

TEST(SHELL_SORT, TEST_SIZE_1569) {
  int rank, num;
  MPI_Comm_rank(MPI_COMM_WORLD, &rank);
  MPI_Comm_size(MPI_COMM_WORLD, &num);

  std::vector<int> vec, vec1, vec2;
  int vec_size = 1569;
  double start, end;

  if (rank == 0) {
    vec = getRandomVector(vec_size);
    vec1 = vec;
    // for (int elem : vec) {
    //  std::cout << elem << " ";
    //}
    // std::cout << std::endl;
    start = MPI_Wtime();
  }
  getParallelShellSort(&vec1);
  if (rank == 0) {
    end = MPI_Wtime();
    double ptime = end - start;
    std::cout << "P time " << ptime << std::endl;
    vec2 = vec;
    start = MPI_Wtime();
    getSequentialShellSort(&vec2);
    end = MPI_Wtime();
    double stime = end - start;
    std::cout << "S time " << stime << std::endl;
    std::cout << "Speedup " << stime / ptime << std::endl;

    std::cout << vec1.size() << " " << vec2.size() << std::endl;
    for (int i = 0; i < vec_size; i++) {
      ASSERT_EQ(vec1[i], vec2[i]);
    }
  }
}

TEST(SHELL_SORT, TEST_SIZE_2048) {
  int rank, num;
  MPI_Comm_rank(MPI_COMM_WORLD, &rank);
  MPI_Comm_size(MPI_COMM_WORLD, &num);

  std::vector<int> vec, vec1, vec2;
  int vec_size = 2048;
  double start, end;

  if (rank == 0) {
    vec = getRandomVector(vec_size);
    vec1 = vec;
    // for (int elem : vec) {
    //  std::cout << elem << " ";
    //}
    // std::cout << std::endl;
    start = MPI_Wtime();
  }
  getParallelShellSort(&vec1);
  if (rank == 0) {
    end = MPI_Wtime();
    double ptime = end - start;
    std::cout << "P time " << ptime << std::endl;
    vec2 = vec;
    start = MPI_Wtime();
    getSequentialShellSort(&vec2);
    end = MPI_Wtime();
    double stime = end - start;
    std::cout << "S time " << stime << std::endl;
    std::cout << "Speedup " << stime / ptime << std::endl;

    std::cout << vec1.size() << " " << vec2.size() << std::endl;
    for (int i = 0; i < vec_size; i++) {
      ASSERT_EQ(vec1[i], vec2[i]);
    }
  }
}

int main(int argc, char** argv) {
  ::testing::InitGoogleTest(&argc, argv);
  MPI_Init(&argc, &argv);

  ::testing::AddGlobalTestEnvironment(new GTestMPIListener::MPIEnvironment);
  ::testing::TestEventListeners& listeners =
      ::testing::UnitTest::GetInstance()->listeners();

  listeners.Release(listeners.default_result_printer());
  listeners.Release(listeners.default_xml_generator());

  listeners.Append(new GTestMPIListener::MPIMinimalistPrinter);
  return RUN_ALL_TESTS();
}
\end{lstlisting}

\end{document}