\documentclass{report}

\usepackage[warn]{mathtext}
\usepackage[T2A]{fontenc}
\usepackage[utf8]{luainputenc}
\usepackage[english, russian]{babel}
\usepackage[pdftex]{hyperref}
\usepackage{tempora}
\usepackage[12pt]{extsizes}
\usepackage{listings}
\usepackage{color}
\usepackage{geometry}
\usepackage{enumitem}
\usepackage{multirow}
\usepackage{graphicx}
\usepackage{indentfirst}
\usepackage{amsmath}

\geometry{a4paper,top=2cm,bottom=2cm,left=2.5cm,right=1.5cm}
\setlength{\parskip}{0.5cm}
\setlist{nolistsep, itemsep=0.3cm,parsep=0pt}

\usepackage{listings}
\lstset{language=C++,
        basicstyle=\footnotesize,
		keywordstyle=\color{blue}\ttfamily,
		stringstyle=\color{red}\ttfamily,
		commentstyle=\color{green}\ttfamily,
		morecomment=[l][\color{red}]{\#}, 
		tabsize=4,
		breaklines=true,
  		breakatwhitespace=true,
  		title=\lstname,       
}

\makeatletter
\renewcommand\@biblabel[1]{#1.\hfil}
\makeatother

\begin{document}

\begin{titlepage}

\begin{center}
Министерство науки и высшего образования Российской Федерации
\end{center}

\begin{center}
Федеральное государственное автономное образовательное учреждение высшего образования \\
Национальный исследовательский Нижегородский государственный университет им. Н.И. Лобачевского
\end{center}

\begin{center}
Институт информационных технологий, математики и механики
\end{center}

\vspace{4em}

\begin{center}
\textbf{\LargeОтчет по лабораторной работе} \\
\end{center}
\begin{center}
\textbf{\Large«Вычисление многомерных интегралов с использованием многошаговой схемы (метод трапеций)»} \\
\end{center}

\vspace{4em}

\newbox{\lbox}
\savebox{\lbox}{\hbox{text}}
\newlength{\maxl}
\setlength{\maxl}{\wd\lbox}
\hfill\parbox{7cm}{
\hspace*{5cm}\hspace*{-5cm}\textbf{Выполнил:} \\ студент группы 381906-3 \\ Куклин А. Е.\\
\\
\hspace*{5cm}\hspace*{-5cm}\textbf{Проверил:}\\ доцент кафедры МОСТ, \\ кандидат технических наук \\ Сысоев А. В.\\
}
\vspace{\fill}

\begin{center} Нижний Новгород \\ 2021 \end{center}

\end{titlepage}

\setcounter{page}{2}

% Содержание
\tableofcontents
\newpage

% Введение
\section*{Введение}
\addcontentsline{toc}{section}{Введение}
\par Численное интегрирование - вычисление значения определенного интеграла с некоторой точностью. На данный момент существует множество различных способов решить интеграл численно. Одним из таких методов является метод трапеций.
\par Метод трапеций, как и некоторые другие методы численного интегрирования, берет за основу определение интеграла. Для вычисления область интегрирования разбивается на отрезки, на каждом отрезке считается значение функции в левой и правой точках, полученное значение делится пополам и умножается на длину отрезка. В случае одномерной функции, использовав данную формулу на каждом отрезке разбиения, мы получим площадь трапеции на каждом интервале. Результатом сложения всех площадей будет площадь подынтегральной функции с некоторой точностью, то есть значение интеграла.
\par Зачастую применение методов численного интегрирования является затратной по времени операцией. Конечно, всё зависит от того, с какой точностью мы хотим получить результат и от системных характеристик ЭВМ. Решение многомерных интегралов, как правило, требует больше времени, чем одномерных. Для решения данной проблемы чаще всего прибегают к методам параллельного программирования. Правильно распараллелив вычисление многомерного интеграла можно добиться значительного ускорения выполнения программы.
\newpage

% Постановка задачи
\section*{Постановка задачи}
\addcontentsline{toc}{section}{Постановка задачи}
\par Необходимо реализовать программу, вычисляющую значение многомерного интеграла методом трапеций. В программе должны быть реализованы последовательный и параллельный алгоритм решения. Для реализации параллельной части программы необходимо использовать библиотеку MPI.
\par Также необходимо проверить корректность работы программы. Для этого нужно задействовать библиотеку модульного тестирования Google Test. При проверке корректности работы программы необходимо убедиться, что параллельная версия работает быстрее последовательной.
\par По окончании работы нужно сделать выводы о том, насколько эффективно распараллеливание подобных алгоритмов. Выводы должны строиться на основании полученных данных о времени работы последовательной и параллельной реализации.
\newpage

% Описание алгоритма
\section*{Описание алгоритма}
\addcontentsline{toc}{section}{Описание алгоритма}
\par Для вычисления определенного интеграла методом трапеций используется следующий алгоритм:
\par Пусть дан двумерный интеграл: $$I = {\int_a^b \int_c^d} f(x^1, x^2) \,dx^1\,dx^2$$
\par Поделим отрезки [a, b] и [c, d] на равные части: таким образом, получим $h_{x^1} = \frac{b - a}{k}$, $h_{x^2} = \frac{d - c}{k}$, где k - произвольное натуральное число, $h_{x^1}$ - высота трапеции по оси x1, $h_{x^2}$ - высота трапеции по оси x2. Отрезки [$x^1_0, x^1_1$], [$x^1_1, x^1_2$], ... , [$x^1_{k-1}, x^1_k$], и [$x^2_0, x^2_1$], [$x^2_1, x^2_2$], ... , [$x^2_{k-1}, x^2_k$] будем называть отрезками разбиения. Вместе они дают плоскость разбиения подынтегральной функции.
\par По всей плоскости разбиения вычислим следующее выражение: $s_m = \frac{f(x^1_{i-1}, x^2_{j-1}) + f(x^1_{i}, x^2_{j})}{2}$, где $i, j = \overline{1, k}$, $m = \overline{1, k^2}$. Полученные значения $s_m$ просуммируем.
\par Найденную сумму умножим на высоты $h_{x^1}$ и $h_{x^2}$, тем самым получим приближенное значение искомого интеграла.
\par Данный алгоритм применим для функций больших размерностей. Для $n$-мерной функции количество областей разбиения будет равным $k^n$, а количество высот трапеций - $n$.
\newpage

% Описание схемы распараллеливания
\section*{Описание схемы распараллеливания}
\addcontentsline{toc}{section}{Описание схемы распараллеливания}
\par Для распараллеливания данного алгоритма, необходимо определить, на каких отрезках разбиения процессы будут вычислять интеграл. Каждый процесс получает количество областей разбиения, равное $\lfloor\frac{segNum}{procNum}\rfloor$, где $segNum$ - количество областей разбиения, $procNum$ - количество запускаемых процессов. Количество областей, равное остатку от деления дополнительно отдается процессу с рангом 0. Затем для каждой переменной подинтегральной функции вычисляется соответствующая высота трапеции. После в каждом процессе запускается подсчет функций в соответствующих областях. Далее полученные суммы редуцируются на нулевой процесс. Полученное значение умножается на все высоты, вычисленные ранее.
\newpage

% Описание программной реализации
\section*{Описание программной реализации}
\addcontentsline{toc}{section}{Описание программной реализации}
\par Программа включает в себя 3 файла: main.cpp, trapezoid\_integral.cpp - исходные файлы, trapezoid\_integral.h - заголовочный файл.
\par В заголовочном файле объявлены все функции, абстрактный класс Func для определения интерфейса подынтегральных функций и наследники класса Func - PiIntegral, FirstIntegral, SecondIntegral, ThirdIntegral. В классе Func объявлены protected поля:
\begin{lstlisting}
std::vector<double> lower_bound, upper_bound;
\end{lstlisting}
\par Поле lower\_bound - вектор, содержащий в себе нижние границы интегрирования всех переменных функции. Поле upper\_bound - вектор, содержащий верхние границы интегрирования всех переменных функции. \par Помимо полей, со спецификатором public в классе Func объявлены следующие функции:
\begin{lstlisting}
  virtual double CalculateFunc(std::vector<double> point) = 0;
  virtual std::vector<double> GetLowerBound();
  virtual std::vector<double> GetUpperBound();
\end{lstlisting}
\par Все перечисленные функции объявлены как виртуальные, чтобы их можно было перегрузить в классах-потомках. CalculateFunc - чистый виртуальный метод. GetLowerBound - виртуальный метод, возвращающий поле lower\_bound, а GetUpperBound - виртуальный метод, возвращающий поле upper\_bound. Конструктор в классе Func не определен, т.к. это абстрактный класс.
\par В классах PiIntegral, FirstIntegral, SecondIntegral, ThirdIntegral объявлены соответствующие конструкторы, а так же объявлена перегрузка функции CalculateFunc.
\par Так же, в файле trapezoid\_integral.h объявлены следующие функции:
\begin{lstlisting}
double CalculateTrapezoidIntegral(Func* func,
                                  const std::vector<double>& lower_bound,
                                  const std::vector<double>& upper_bound,
                                  std::size_t segments_count,
                                  std::size_t f_point, std::size_t l_point);
double TrapezoidIntegralSequential(Func* func,
                                   const std::vector<double>& lower_bound,
                                   const std::vector<double>& upper_bound,
                                   std::size_t segments_count);
double TrapezoidIntegralParallel(Func* func,
                                 const std::vector<double>& lower_bound,
                                 const std::vector<double>& upper_bound,
                                 std::size_t segments_count);
\end{lstlisting}
Функции TrapezoidIntegralSequential и TrapezoidIntegralParallel принимают на вход указатель на объект класса Func, вектор нижних границ интегрирования lower\_bound, вектор верхних границ интегрирования upper\_bound и количество отрезков, на которые мы разбиваем интегрирование segment\_count. Обе функции возвратят значение интеграла от функции func.
\par Функция CalculateTrapezoidIntegral принимает на вход указатель на объект класса Func, вектор lower\_bound - вектор нижних границ интегрирования, upper\_bound - вектор верхник границ интегрирования, segment\_count - количество отрезков, на которые мы разбиваем область интегрирования, f\_point - номер начала отрезка интегрирования, l\_point - номер конца отрезка интегрирования.
\par Реализации объявленных в trapezoid\_integral.h файле функций и методов находятся в файле trapezoid\_integral.cpp. Функции TrapezoidIntegralSequential и TrapezoidIntegralParallel определяют параметры f\_point и l\_point для функции CalculateTrapezoidIntegral. Так как\break TrapezoidIntegralSequential выполняет последовательное вычисление интеграла, то будет вызвана функция CalculateTrapezoidIntegral для подсчета интеграла в единственном процессе на всех отрезках разбиения. Функция TrapezoidIntegralParallel сначала определит, какое количество отрезков разбиения отдать каждому процессу, а затем вызовет во всех процессах функцию CalculateTrapezoidIntegral с соответствующими отрезками. Полученные на каждом процессе результаты редуцируются на нулевой процесс с примением операции MPI\_SUM.
\par Файл main.cpp содержит реализацию набора unit-тестов, реализованных с помощью библиотеки Google Test. При запуске программы, main запустит тестирование.
\newpage

% Подтверждение корректности
\section*{Подтверждение корректности}
\addcontentsline{toc}{section}{Подтверждение корректности}
\par Корректность работы программы была проверена с помощью инструмента модульного тестирования Google Test. В тестах сравниваются полученные при выполнении последовательного и параллельного вычисления значения интегралов. Для тестирования были выбраны следующие подынтегральные функции:
\begin{itemize}
  \item $f(x, y) = x^2 + y^2; x, y \in [0, 1]$
  \item $f(x, y) = xy; x, y \in [0, 2]$
  \item $f(x, y, z) = sin(x) + cos(y) + sin(z); x, y, z \in [-1, 1]$
\end{itemize}
\par Помимо проверки корректности полученного результата, реализации проверялись на скорость работы. Тесты показали, что параллельный алгоритм работает быстрее на всех представленных функциях.
\newpage

% Результаты экспериментов
\section*{Результаты экспериментов}
\addcontentsline{toc}{section}{Результаты экспериментов}
Скорость выполнения программы во многом зависит от архитектуры ЭВМ. Мною был использован ПК со следующими характеристиками:
\begin{itemize}
\item Процессор: Intel(R) Core(TM) i5-4570 CPU @ 3.20GHz, 3201 МГц, ядер: 4, логических процессоров: 4;
\item Оперативная память (RAM): 8,00 ГБ;
\item Операционная система: Майкрософт Windows 10 Pro
\end{itemize}
\par Эксперементы проведены на разных количествах процессов (proc\_count), представлено время работы последовательного алгоритма - time\_seq (в секундах), время работы параллельного алгоритма - time\_paral (в секундах) и ускорение (boost), которое считается по формуле boost = $\frac{time\_seq}{time\_paral}$.

\par В следующих таблицах представлены результаты:.

\begin{table}[!h]
\centering
\begin{tabular}{| p{2cm} | p{3cm} | p{3cm} | p{2cm} |}
\hline
proc\_count & time\_seq (s) & time\_paral (s) & boost  \\[5pt]
\hline
1        & 1.63253        & 1.63281      & 0.999831  \\
2        & 1.6298         & 0.809603     & 2.01308   \\
3        & 1.65863        & 0.54084      & 3.06677   \\
4        & 1.62344        & 0.432001     & 3.75795   \\
6        & 1.62081        & 0.443871     & 3.65153   \\
8        & 1.62303        & 0.444979     & 3.64743	 \\
12       & 1.62581        & 0.467139     & 3.48036	 \\
\hline
\end{tabular}
\caption{Тестовая функция: $f(x, y) = x^2 + y^2; x, y \in [0, 1]$, количество отрезков разбинеия 1000000}
\end{table}

\begin{table}[!h]
\centering
\begin{tabular}{| p{2cm} | p{3cm} | p{3cm} | p{2cm} |}
\hline
proc\_count & time\_seq (s) & time\_paral (s) & boost  \\[5pt]
\hline
1        & 1.66608        & 1.69341      & 0.983861  \\
2        & 1.60445        & 0.847927     & 1.8922    \\
3        & 1.62014        & 0.593598     & 2.72935   \\
4        & 1.60603        & 0.435211     & 3.69024   \\
6        & 1.6032         & 0.441152     & 3.63411	 \\
8        & 1.60158        & 0.450639     & 3.55402	 \\
12       & 1.60857        & 0.45362      & 3.54607   \\
\hline
\end{tabular}
\caption{Тестовая функция: $f(x, y) = x*y; x, y \in [0, 2]$, количество отрезков разбинеия 1000000}
\end{table}

\newpage

% Выводы из результатов экспериментов
\section*{Выводы из результатов экспериментов}
\addcontentsline{toc}{section}{Выводы из результатов экспериментов}
\par Исходя из приведенных результатов, можно сказать, что в общем случае увеличение количества процессов приводит к ускорению программы. Т.к. количество потоков процессора равно 4, то ускорение на количестве процессов большим 4 будет уменьшаться. Это происходит из-за конкуренции процессов за вычислительные ресурсы процессора: чем больше процессов будут участвовать, тем больше им придется конкурировать. Однако, существуют ситуации, когда ускорение превышает число процессов, участвующих в программе. Это связано с использованием кэш-памяти некоторыми процессами, что позволяет достичь ещё большего ускорения.

\newpage

% Заключение
\section*{Заключение}
\addcontentsline{toc}{section}{Заключение}
\par В ходе данной задачи были реализованы последовательный и параллельный алгоритм вычисления многомерного интеграла методом трапеции, что позволило понять принципы работы параллельного программирования и библиотеки MPI. В результате эксперементов выяснилось, что распараллеливание подобных алгоритмов зачастую сильно ускоряет работу программы, что говорит о важности и актуальности проделанной работы.
\newpage

% Литература
\section*{Литература}
\addcontentsline{toc}{section}{Литература}
\begin{enumerate}
\item Самарский А. А. Введение в численные методы: учебное пособие. М: Издательство «Лань», Санкт-Петербург, 2005. - 137 с.
\item Wikipedia - электронный ресурс. URL: \url{https://ru.wikipedia.org/wiki/Message_Passing_Interface} (дата обращения: 02.12.2021)
\end{enumerate} 
\newpage

% Приложение
\section*{Приложение}
\addcontentsline{toc}{section}{Приложение}
\par trapezoid\_integral.h
\begin{lstlisting}
// Copyright 2021 Kuklin Andrey
#ifndef MODULES_TASK_3_KUKLIN_A_TRAPEZOID_INTEGRAL_TRAPEZOID_INTEGRAL_H_
#define MODULES_TASK_3_KUKLIN_A_TRAPEZOID_INTEGRAL_TRAPEZOID_INTEGRAL_H_

#include <vector>
#include <cmath>

class Func {
 protected:
  std::vector<double> lower_bound, upper_bound;

 public:
  virtual double CalculateFunc(std::vector<double> point) = 0;
  virtual std::vector<double> GetLowerBound();
  virtual std::vector<double> GetUpperBound();
};

class PiIntegral : public Func {
 public:
  PiIntegral();
  double CalculateFunc(std::vector<double> point);
};

class FirstIntegral : public Func {
 public:
  FirstIntegral();
  double CalculateFunc(std::vector<double> point);
};

class SecondIntegral : public Func {
 public:
  SecondIntegral();
  double CalculateFunc(std::vector<double> point);
};

class ThirdIntegral : public Func {
 public:
  ThirdIntegral();
  double CalculateFunc(std::vector<double> point);
};

double CalculateTrapezoidIntegral(Func* func,
                                  const std::vector<double>& lower_bound,
                                  const std::vector<double>& upper_bound,
                                  std::size_t segments_count,
                                  std::size_t f_point, std::size_t l_point);

double TrapezoidIntegralSequential(Func* func,
                                   const std::vector<double>& lower_bound,
                                   const std::vector<double>& upper_bound,
                                   std::size_t segments_count);

double TrapezoidIntegralParallel(Func* func,
                                 const std::vector<double>& lower_bound,
                                 const std::vector<double>& upper_bound,
                                 std::size_t segments_count);

#endif  //  MODULES_TASK_3_KUKLIN_A_TRAPEZOID_INTEGRAL_TRAPEZOID_INTEGRAL_H_
\end{lstlisting}
\par trapezoid\_integral.cpp
\begin{lstlisting}
// Copyright 2021 Kuklin Andrey
#include <mpi.h>
#include <vector>
#include "../../../modules/task_3/kuklin_a_trapezoid_integral/trapezoid_integral.h"

std::vector<double> Func::GetLowerBound() { return lower_bound; }

std::vector<double> Func::GetUpperBound() { return upper_bound; }

PiIntegral::PiIntegral() {
  lower_bound = {0.0};
  upper_bound = {1.0};
}

double PiIntegral::CalculateFunc(std::vector<double> point) {
  return 4.0 / (1.0 + point[0] * point[0]);
}

FirstIntegral::FirstIntegral() {
  lower_bound = {0.0, 0.0};
  upper_bound = {1.0, 1.0};
}

double FirstIntegral::CalculateFunc(std::vector<double> point) {
  return point[0] * point[0] + point[1] * point[1];
}

SecondIntegral::SecondIntegral() {
  lower_bound = {0.0, 0.0};
  upper_bound = {2.0, 2.0};
}

double SecondIntegral::CalculateFunc(std::vector<double> point) {
  return point[0] * point[1];
}

ThirdIntegral::ThirdIntegral() {
  lower_bound = {-1.0, -1.0, -1.0};
  upper_bound = {1.0, 1.0, 1.0};
}

double ThirdIntegral::CalculateFunc(std::vector<double> point) {
  return sin(point[0]) + cos(point[1]) + sin(point[2]);
}

double CalculateTrapezoidIntegral(Func* func,
                                  const std::vector<double>& lower_bound,
                                  const std::vector<double>& upper_bound,
                                  std::size_t segments_count,
                                  std::size_t f_point, std::size_t l_point) {
  std::size_t variables_count = lower_bound.size();
  std::vector<double> height(variables_count);
  double res = 0.0;

  std::vector<double> curr_seg(variables_count);
  std::size_t tmp_p = f_point;
  for (std::size_t i = 0; i < variables_count; i++) {
    curr_seg[i] = tmp_p % segments_count;
    tmp_p /= segments_count;
  }

  for (std::size_t i = 0; i < variables_count; i++)
    height[i] = (upper_bound[i] - lower_bound[i]) / segments_count;

  std::vector<double> prev_point(variables_count), curr_point(variables_count);
  for (std::size_t i = f_point; i < l_point; i++) {
    for (std::size_t j = 0; j < variables_count; j++) {
      prev_point[j] = lower_bound[j] + curr_seg[j] * height[j];
      curr_point[j] = prev_point[j] + height[j];
    }
    res +=
        (func->CalculateFunc(prev_point) + func->CalculateFunc(curr_point)) / 2;
    curr_seg[0] += 1;
    for (std::size_t j = 0; j < variables_count - 1; j++) {
      if (curr_seg[j] == segments_count) {
        curr_seg[j] = 0;
        ++curr_seg[j + 1];
      } else {
        break;
      }
    }
  }
  for (std::size_t i = 0; i < variables_count; i++) res *= height[i];

  return res;
}

double TrapezoidIntegralSequential(Func* func,
                                   const std::vector<double>& lower_bound,
                                   const std::vector<double>& upper_bound,
                                   std::size_t segments_count) {
  std::size_t points_count = 1, variables_count = lower_bound.size();
  for (std::size_t i = 0; i < variables_count; i++)
    points_count *= segments_count;
  return CalculateTrapezoidIntegral(func, lower_bound, upper_bound,
                                    segments_count, 0, points_count);
}

double TrapezoidIntegralParallel(Func* func,
                                 const std::vector<double>& lower_bound,
                                 const std::vector<double>& upper_bound,
                                 std::size_t segments_count) {
  int procSize, procRank;
  MPI_Comm_size(MPI_COMM_WORLD, &procSize);
  MPI_Comm_rank(MPI_COMM_WORLD, &procRank);

  std::size_t variables_count = lower_bound.size(), points_count = 1;

  for (std::size_t i = 0; i < variables_count; i++)
    points_count *= segments_count;

  std::size_t local_segment_size = points_count / procSize;
  std::size_t remenian_size = points_count % procSize;
  std::size_t f_point = procRank * local_segment_size + remenian_size;
  std::size_t l_point = (procRank + 1) * local_segment_size + remenian_size;

  if (procRank == 0) f_point = 0;

  double local_integral = CalculateTrapezoidIntegral(
      func, lower_bound, upper_bound, segments_count, f_point, l_point);

  double res = 0.0;
  MPI_Reduce(&local_integral, &res, 1, MPI_DOUBLE, MPI_SUM, 0, MPI_COMM_WORLD);

  return res;
}
\end{lstlisting}
\par main.cpp
\begin{lstlisting}
// Copyright 2021 Kuklin Andrey
#include <gtest/gtest.h>
#include <iostream>
#include "./trapezoid_integral.h"
#include <gtest-mpi-listener.hpp>

TEST(Integral_Sequential, can_solve_integral_sequential) {
  Func *f = new PiIntegral();
  ASSERT_NO_THROW(TrapezoidIntegralSequential(f, f->GetLowerBound(),
                                              f->GetUpperBound(), 10));
}

TEST(Integral_Sequential, can_solve_integral_parallel) {
  Func *f = new PiIntegral();
  ASSERT_NO_THROW(
      TrapezoidIntegralParallel(f, f->GetLowerBound(), f->GetUpperBound(), 10));
}

TEST(Integral_Sequential, seq_parall_is_eqv) {
  Func *f = new PiIntegral();
  int procRank;
  MPI_Comm_rank(MPI_COMM_WORLD, &procRank);
  double res_sequal = TrapezoidIntegralSequential(f, f->GetLowerBound(),
                                                  f->GetUpperBound(), 100);
  double res_paral =
      TrapezoidIntegralParallel(f, f->GetLowerBound(), f->GetUpperBound(), 100);

  if (procRank == 0) {
    ASSERT_TRUE(fabs(res_paral - res_sequal) < 0.01);
  }
}

TEST(Integral_Sequential, first_integral) {
  Func *f = new FirstIntegral();
  int procRank;
  MPI_Comm_rank(MPI_COMM_WORLD, &procRank);
  double res_sequal = TrapezoidIntegralSequential(f, f->GetLowerBound(),
                                                  f->GetUpperBound(), 150);
  double res_paral =
      TrapezoidIntegralParallel(f, f->GetLowerBound(), f->GetUpperBound(), 150);

  if (procRank == 0) {
    ASSERT_TRUE(fabs(res_paral - res_sequal) < 0.01);
  }
}

TEST(Integral_Sequential, second_integral) {
  Func *f = new SecondIntegral();
  int procRank;
  MPI_Comm_rank(MPI_COMM_WORLD, &procRank);
  double res_sequal = TrapezoidIntegralSequential(f, f->GetLowerBound(),
                                                  f->GetUpperBound(), 200);
  double res_paral =
      TrapezoidIntegralParallel(f, f->GetLowerBound(), f->GetUpperBound(), 200);

  if (procRank == 0) {
    ASSERT_TRUE(fabs(res_paral - res_sequal) < 0.01);
  }
}

TEST(Integral_Sequential, third_integral) {
  Func *f = new ThirdIntegral();
  int procRank;
  MPI_Comm_rank(MPI_COMM_WORLD, &procRank);

  double res_sequal = 0.0;
  if (procRank == 0) {
    res_sequal = TrapezoidIntegralSequential(f, f->GetLowerBound(),
                                             f->GetUpperBound(), 100);
  }

  MPI_Barrier(MPI_COMM_WORLD);
  double res_paral =
      TrapezoidIntegralParallel(f, f->GetLowerBound(), f->GetUpperBound(), 100);

  if (procRank == 0) {
    ASSERT_TRUE(fabs(res_paral - res_sequal) < 0.01);
  }
}

TEST(Integral_Sequential, time_test) {
  Func *f = new ThirdIntegral();
  int procRank;
  double start_time, end_time, seq_time, par_time;
  MPI_Comm_rank(MPI_COMM_WORLD, &procRank);

  double res_sequal = 0.0;
  if (procRank == 0) {
    start_time = MPI_Wtime();
    res_sequal = TrapezoidIntegralSequential(f, f->GetLowerBound(),
                                             f->GetUpperBound(), 100);
    end_time = MPI_Wtime();
    seq_time = end_time - start_time;
    std::cout << "time seq: " << seq_time << "\n";
    start_time = MPI_Wtime();
  }

  MPI_Barrier(MPI_COMM_WORLD);
  double res_paral =
      TrapezoidIntegralParallel(f, f->GetLowerBound(), f->GetUpperBound(), 100);

  if (procRank == 0) {
    end_time = MPI_Wtime();
    par_time = end_time - start_time;
    std::cout << "time par: " << par_time << "\n";
    std::cout << "parall value: " << seq_time / par_time << "\n";
    ASSERT_TRUE(fabs(res_paral - res_sequal) < 0.01);
  }
}

int main(int argc, char **argv) {
  ::testing::InitGoogleTest(&argc, argv);
  MPI_Init(&argc, &argv);

  ::testing::AddGlobalTestEnvironment(new GTestMPIListener::MPIEnvironment);
  ::testing::TestEventListeners &listener =
      ::testing::UnitTest::GetInstance()->listeners();

  listener.Release(listener.default_result_printer());
  listener.Release(listener.default_xml_generator());

  listener.Append(new GTestMPIListener::MPIMinimalistPrinter);
  return RUN_ALL_TESTS();
}
\end{lstlisting}

\end{document}