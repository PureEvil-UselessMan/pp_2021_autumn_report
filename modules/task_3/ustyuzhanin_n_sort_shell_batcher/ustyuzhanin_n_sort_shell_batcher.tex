\documentclass{report}

\usepackage[warn]{mathtext}
\usepackage[T2A]{fontenc}
\usepackage[utf8]{luainputenc}
\usepackage[english, russian]{babel}
\usepackage[pdftex]{hyperref}
\usepackage{tempora}
\usepackage[12pt]{extsizes}
\usepackage{listings}
\usepackage{color}
\usepackage{geometry}
\usepackage{enumitem}
\usepackage{multirow}
\usepackage{graphicx}
\usepackage{indentfirst}
\usepackage{amsmath}

\geometry{a4paper,top=2cm,bottom=2cm,left=2.5cm,right=1.5cm}
\setlength{\parskip}{0.5cm}
\setlist{nolistsep, itemsep=0.3cm,parsep=0pt}

\usepackage{listings}
\lstset{language=C++,
        basicstyle=\footnotesize,
		keywordstyle=\color{blue}\ttfamily,
		stringstyle=\color{red}\ttfamily,
		commentstyle=\color{green}\ttfamily,
		morecomment=[l][\color{red}]{\#}, 
		tabsize=4,
		breaklines=true,
  		breakatwhitespace=true,
  		title=\lstname,       
}

\makeatletter
\renewcommand\@biblabel[1]{#1.\hfil}
\makeatother

\begin{document}

\begin{titlepage}

\begin{center}
Министерство науки и высшего образования Российской Федерации
\end{center}

\begin{center}
Федеральное государственное автономное образовательное учреждение высшего образования \\
Национальный исследовательский Нижегородский государственный университет им. Н.И. Лобачевского
\end{center}

\begin{center}
Институт информационных технологий, математики и механики
\end{center}

\vspace{4em}

\begin{center}
\textbf{\LargeОтчет по лабораторной работе} \\
\end{center}
\begin{center}
\textbf{\Large«Сортировка Шелла с четно-нечетным слиянием Бэтчера.»} \\
\end{center}

\vspace{4em}

\newbox{\lbox}
\savebox{\lbox}{\hbox{text}}
\newlength{\maxl}
\setlength{\maxl}{\wd\lbox}
\hfill\parbox{7cm}{
\hspace*{5cm}\hspace*{-5cm}\textbf{Выполнил:} \\ студент группы 381906-1 \\ Устюжанин Н. В.\\
\\
\hspace*{5cm}\hspace*{-5cm}\textbf{Проверил:}\\ доцент кафедры МОСТ, \\ кандидат технических наук \\ Сысоев А. В.\\
}
\vspace{\fill}

\begin{center} Нижний Новгород \\ 2021 \end{center}

\end{titlepage}

\setcounter{page}{2}

% Содержание
\tableofcontents
\newpage

% Введение
\section*{Введение}
\addcontentsline{toc}{section}{Введение}
\par Алгоритм сортировки — это алгоритм для упорядочивания элементов в массиве. Первые прототипы современных методов сортировки появились уже в XIX веке. В данной лабораторной работе требуется реализовать алгоритм сортировки Шелла и четно-нечетного слияния Бетчера.
\newpage

% Постановка задачи
\section*{Постановка задачи}
\addcontentsline{toc}{section}{Постановка задачи}
\par В данной лабораторной работе требуется реализовать последовательный алгоритм сортировки Шелла и параллельный алгоритм сортировки Шелла с четно-нечетным слиянием Бетчера, используя MPI, провести вычислительные эксперименты для сравнения времени работы алгоритмов, используя библиотеку для модульного тестирования  Google Test, сделать выводы об эффективности реализованных алгоритмов.
\newpage

% Описание алгоритма
\section*{Описание алгоритма}
\addcontentsline{toc}{section}{Описание алгоритма}
\par Сортировка Шелла — алгоритм сортировки, являющийся усовершенствованным вариантом сортировки вставками. Идея метода Шелла состоит в сравнении элементов, стоящих не только рядом, но и на определённом расстоянии друг от друга. Иными словами — это сортировка вставками с предварительными «грубыми» проходами.
\par При сортировке Шелла сначала сравниваются и сортируются между собой значения, стоящие один от другого на некотором расстоянии $d$. После этого процедура повторяется для некоторых меньших значений $d$, а завершается сортировка Шелла упорядочиванием элементов при $d = 1$ (то есть обычной сортировкой вставками). Эффективность сортировки Шелла в определённых случаях обеспечивается тем, что элементы «быстрее» встают на свои места (в простых методах сортировки, например, пузырьковой, каждая перестановка двух элементов уменьшает количество инверсий в списке максимум на 1, а при сортировке Шелла это число может быть больше).

Невзирая на то, что сортировка Шелла во многих случаях медленнее, чем быстрая сортировка, она имеет ряд преимуществ:
\begin{itemize}
    \item отсутствие потребности в памяти под стек;
    \item отсутствие деградации при неудачных наборах данных — быстрая сортировка легко деградирует до $O(n^2$), что хуже, чем худшее гарантированное время для сортировки Шелла.
\end{itemize}
Сложность алгоритма:
\begin{itemize}
    \item Худшее время: $O(n^2$)
    \item Лучшее время: $O(nlog^2n)$
    \item Среднее время: зависит от выбранных шагов
\end{itemize}
\par Чётно-нечётное слияние Бэтчера заключается в том, что два упорядоченных массива, которые необходимо слить, разделяются на чётные и нечётные элементы. Такое слияние может быть выполнено параллельно.
Чтобы массив стал окончательно отсортированным, достаточно сравнить
пары элементов, стоящие на нечётной и чётной позициях. Первый и последний элементы массива проверять не надо, т.к. они являются минимальным и максимальным элементов массивов.
Чётно-нечётное слияние Бэтчера позволяет задействовать 2 потока при
слиянии двух упорядоченных массивов. В этом случае слияние n массивов
могут выполнять n параллельных потоков. На следующем шаге слияние $n/2$
полученных массивов будут выполнять $n/2$ потоков и т.д. На последнем
шаге два массива будут сливать 2 потока.
\newpage
% Описание схемы распараллеливания
\section*{Описание схемы распараллеливания}
\addcontentsline{toc}{section}{Описание схемы распараллеливания}
\par Процесс с рангом 0 генерирует случайный массив исходных значений, разбивает его на $n$ частей и отправляет $n$-му процессу $n$-ю часть массива. Каждый процесс сортирует свой часть сортировкой Шелла и отправляет $0$-му процессу уже отсортированный фрагмент массива, после чего вызывается алгоритм четно-нечетного слияния Бетчера, который объединяет все фрагменты в один. Этот массив и будет результатом работы алгоритма.
\par Замечание: количество процессов всегда должно быть степенью двойки; размер сортируемого массива должен делиться на количество процессов без остатка.

\newpage

% Описание программной реализации
\section*{Описание программной реализации}
\addcontentsline{toc}{section}{Описание программной реализации}
Программа состоит из заголовочного файла sort\_shell\_batcher.h и двух файлов исходного кода sort\_shell\_batcher.cpp и main.cpp.
\par В заголовочном файле находятся прототипы функций для последовательного и параллельного выполнения
\par Функция для последовательного алгоритма:
\begin{lstlisting}
void shellSort(vector<int> *arr);
\end{lstlisting}
Параметром функции является указатель на std::vector, который требуется отсортировать.
\par Функция для параллельного алгоритма:
\begin{lstlisting}
vector<int> shellSortMPI(vector<int> arr, int size);
\end{lstlisting}
Первым параметром является std::vector, который требуется отсортировать, вторым - его размер.
\par В файле sort\_shell\_batcher.cpp содержится реализация функций, объявленных в заголовочном файле. В файле исходного кода main.cpp содержатся тесты для проверки корректности программы.
\newpage

% Подтверждение корректности
\section*{Подтверждение корректности}
\addcontentsline{toc}{section}{Подтверждение корректности}
Для подтверждения корректности работы данной программы с помощью библиотеки для модульного тестирования  Google Test мной было разработано 6 тестов. 
\begin{itemize}
    \item 1-й тест проверяет корректность работы последовательного алгоритма сортировки Шелла
    \item 2-й тест проверяет корректность работы последовательного алгоритма четно-нечетного слияния Бетчера
    \item 3-й тест проверяет корректность работы параллельного алгоритма четно-нечетного слияния Бетчера
    \item 4-й тест проверяет корректность работы параллельного алгоритма сортировки Шелла
    \item 5-й тест проверяет корректность работы при огромном наборе входных данных
    \item 6-й тест проверяет корректность работы при минимальном наборе входных данных
\end{itemize}
\par Успешное прохождение всех тестов подтверждает корректность работы программы.
\newpage

% Результаты экспериментов
\section*{Результаты экспериментов}
\addcontentsline{toc}{section}{Результаты экспериментов}
Вычислительные эксперименты для оценки эффективности работы параллельного алгоритма проводились на ПК со следующими характеристиками:
\begin{itemize}
\item Процессор: Intel Pentium 4415U, 2.30 GHz, 4 ядра
\item Оперативная память: 4 ГБ (DDR4), 3200 МГц;
\item Операционная система: Windows 10 Home.
\end{itemize}

\par Эксперименты проводились на различном числе процессов для сортировки массива размером $size=2^{22}$.

\par Результаты экспериментов представлены в Таблице 1.

\begin{table}[!h]
\caption{Результаты вычислительных экспериментов}
\centering
\begin{tabular}{| p{2cm} | p{3cm} | p{4cm} |}
\hline
Количество процессов & Время работы последовательного алгоритма (в секундах) & Время работы параллельного алгоритма (в секундах) \\[5pt]
\hline
2        &  5.24473       &  2.95523       \\
4        &  5.33771        & 2.21771    \\
8        &  5.81569     & 2.04935      \\
16        &  5.01022       &  1.56785       \\
32        &  4.88608       & 1.18366       \\
64        & 5.16532       & 1.12238    \\
\hline
\end{tabular}
\end{table}

\newpage

% Выводы из результатов экспериментов
\section*{Выводы из результатов экспериментов}
\addcontentsline{toc}{section}{Выводы из результатов экспериментов}
Из данных, полученных в результате экспериментов (см. Таблицу 1), можно сделать вывод, что параллельный алгоритм работает быстрее, чем последовательный. Можно заметить, что с увеличением числа процессов растёт ускорение. Однако, начиная с некоторого количества процессов, ускорение начинает немного уменьшаться. Это связано с тем, что псевдослучайные числа генерируются только в одном процессе, а затем частями передаются из этого процесса во все остальные процессы, соответственно, чем больше процессом, тем больше времени уходит на распределение данных.
\newpage

% Заключение
\section*{Заключение}
\addcontentsline{toc}{section}{Заключение}
Таким образом, в рамках данной лабораторной работы были разработаны последовательный и параллельный алгоритмы сортировки Шелла. Проведенные тесты показали корректность реализованной программы, а проведенные эксперименты доказали эффективность параллельного алгоритма по сравнению с последовательным.
\newpage

% Литература
\section*{Литература}
\addcontentsline{toc}{section}{Литература}
\begin{enumerate}
\item \href {https://ru.wikipedia.org/wiki/%D0%90%D0%BB%D0%B3%D0%BE%D1%80%D0%B8%D1%82%D0%BC_%D1%81%D0%BE%D1%80%D1%82%D0%B8%D1%80%D0%BE%D0%B2%D0%BA%D0%B8}{Википедия-Алгоритм сортировки}
\item \href {https://ru.wikipedia.org/wiki/%D0%A1%D0%BE%D1%80%D1%82%D0%B8%D1%80%D0%BE%D0%B2%D0%BA%D0%B0_%D0%A8%D0%B5%D0%BB%D0%BB%D0%B0}{Википедия-Сортировка Шелла}
\item \href {https://ru.wikipedia.org/wiki/Message_Passing_Interface}{Википедия-MPI}
\item \href {http://hpc-education.unn.ru/files/courses/XeonPhi/Lab07.pdf}{Сиднев А.А., Сысоев А.В., Мееров И.Б. -Оптимизация вычислительно трудоемкого программного модуля для архитектуры Intel Xeon Phi. Линейные сортировки}
\end{enumerate} 
\newpage

% Приложение
\section*{Приложение}
\addcontentsline{toc}{section}{Приложение}

\href {https://github.com/allnes/pp_2021_autumn/tree/master/modules/task_3/ustyuzhanin_n_sort_shell_batcher}{Исходные коды}

\end{document}